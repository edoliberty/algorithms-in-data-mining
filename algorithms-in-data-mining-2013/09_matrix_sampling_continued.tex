\documentclass{article}
\usepackage{algorithms_in_data_mining}
\begin{document}

\lecture{9}{Matrix approximation continued}{Edo Liberty}


\subsection*{Recap} Last class was dedicated to approximating matrices by sampling individual entries from them.
We used the following useful concentration result for sums of random matrices.

\begin{lemma}[Matrix Bernstein Inequality \cite{CandesRechtMatrixBernsteinInequality}]
Let $X_1,\ldots, X_s$ be independent $m \times n$ matrix valued random variables such that 
\[
\forall _{k \in [s]} \mbox{\;\;\;\;} \E[X_k] = 0 \mbox{\;\;\;\;and\;\;\;\;} \|X_k\| \le R
\]
Set $\sigma^2 = \max\{\|\sum_k \E[X_k X_k^T]\|, \|\sum_k \E[X_k^T X_k]\|\}$ then
\[
\Pr[\|\sum X_{k}\| > t] \le (m+n)e^{-\frac{t^2}{\sigma^2 + Rt/3}}
\]
\end{lemma}
%
\noindent We obtained an approximate sampled matrix $B$ which was sparser than $A$.
For example, for a matrix $A$ containing values in $\{1,0,-1\}$ it sufficed that $B$ contains only $s$ entries and 
\[
s \in O\left(\frac{nr \log(n/\delta)}{\eps^2}\right).
\]
Moreover, we got that $\|A - B\| \le \eps \|A\|$. 
We also claimed that we can compute the PCA projection of $B$, $P_k^B$ instead of that of $A$, $P_k$ and still have:
 \[
 \sigma_{k+1} =  \|A - P_k A\|  \le \|A - P_k^B A\| \le \sigma_{k+1} + \eps\|A\|
 \]

\noindent In this class we will reduce the amount of space needed to be independent of $n$. We will do this by approximating $AA^T$ directly.
Here we follow the ideas in \cite{rvSamplingFromLargeMatrices2007} and give a much simpler proof which, unfortunately, obtains slightly worse bounds.

\subsection*{PCA with an approximate covariance matrix}
Here we see that approximating $BB^T \sim AA^T$ is sufficient in order to compute an approximate PCA projection (Lemma 3.8 in \cite{rvSamplingFromLargeMatrices2007}).
Let $P_k^B$ denote the projection on the top $k$ left singular values of $B$.  
\begin{eqnarray}
\|A - P_k^B A\|^2 &\le& \sup_{x, \|x\|=1} \|xA - xP_k^B A\|^2 \\
&=& \sup_{x \in null(P_k^B), \|x\|=1} \|xA\|^2 \\
&=& \sup_{x \in null(P_k^B), \|x\|=1} \|xAA^T\| \\
&\le& \sup_{x \in null(P_k^B), \|x\|=1} \|x(AA^T-BB^T)\|  + \|xBB^T\|^2\\
&\le& \|AA^T - BB^T\| + \sigma_{k+1}\E(BB^T)\\
&\le& 2\|AA^T - BB^T\| + \sigma_{k+1}(AA^T)
\end{eqnarray}
The last transition is due to the fact that $ \sigma_{k+1}(BB^2) \le  \sigma_{k+1}(AA^T) + \|AA^T - BB^T\|$.
By taking the square root and recalling that $\sigma_{k+1}(AA^T) = \sigma_{k+1}^2(A)  \dot{=}  \sigma_{k+1}^2$ we get:
\[
\|A - P_k^B A\| \le \sqrt{\sigma_{k+1}^2 + 2\|AA^T - BB^T\|} \le \sigma_{k+1} +\sqrt{2\|AA^T - BB^T\|}
\]
Therefore, we have that 
\[
\sqrt{2\|AA^T - BB^T\|} \le \eps\|A\| \Longrightarrow \|A - P_k^BA\| \le \sigma_{k+1}+\eps\|A\|
\]

\subsection*{Column subset selection by sampling}
From this point on, our goal is to find a matrix $B$ such that $\|AA^T - BB^T\|$ is small and $B$ is as sparse or small as possible.
Note that 
\[
AA^T = \sum_{j=1}^{n}A_jA_j^2
\]
where we denote by $A_j$ the $j$'th column of the matrix $A$.
Let 
\[
B \leftarrow A_j/\sqrt{p_j} \mbox{ with probability } p_j
\]
Computing the expectation of $BB^T$ we have 
\[
\E[BB^T] = \sum_{j=1}^{n} p_j (A_j/\sqrt{p_j})(A_j/\sqrt{p_j})^T = \sum_{j=1}^{n} A_jA_j^T = AA^T
\]
Clearly, we cannot hope to approximate the matrix $A$ with only one column. 
We therefore define $B$ to be $s$ such sampled columns from $A$ side by side.
\[
B = \frac{1}{\sqrt{s}}[B_1 | \ldots | B_s]
\]
Just to clarify, $B$ is an $m \times s$ matrix containing $s$ columns from $A$ (rescaled) and $B_k \leftarrow A_j/\sqrt{p_j}$ with probability $p_j$.
Computing the expectation of $BB^T$ we get that 
\[
\E[BB^T] = \sum_{k=1}^{s} \E[\frac{1}{s}B_kB_k^T] = \E[B_kB_k^T] = AA^T
\] 

We are now ready to use the matrix Bernstein inequality above: 
\[
\|BB^T - AA^T\| = \|\sum_{k=1}^{s}  \frac{1}{s}(B_kB_k^T - AA^T)\| = \|\sum_{k=1}^{s}  X_k\|
\]
To make things simpler we pick $p_j = \|A_j\|^2/\|A\|_F^2$. In words, the columns are picked with probability proportional to their squared $2$ norm.
\[
R = \max \|X_k\| \le \max_{j=1}^{n} \|  \frac{1}{s}A_jA_j^T/p_j\| +  \|\frac{1}{s}AA^t\| = \frac{1}{s}\|A\|_F^2 + \frac{1}{s}\|A\|_2^2
\]
\begin{eqnarray}
\sigma^2 &=& \|\sum_k \E[X_k X_k^T]\| = \frac{1}{s}\|\E[B_k B_k^TB_k B_k^T - AA^TAA^T]\| \\
&\le& \frac{1}{s}\|\sum_{j=1}^{n}p_jA_j A_j^TA_j A_j^T/p_j^2\|+ \frac{1}{s}\|A\|_2^4 = \frac{1}{s} \|A\|_F^2\|A\|_2^2 + \frac{1}{s} \|A\|_2^4
\end{eqnarray}

Plugging both expressions into the matrix Chernoff bound above we get that
\[
\Pr[\|BB^T - AA^T\| > \eps^2\|A\|_2^2/2] \le 2me^{-\frac{s\eps^4\|A\|_{2}^{4}/4}{\|A\|_F^2\|A\|_2^2 + \|A\|_2^4 + \eps^2\|A\|_2^2\|A\|_F^2/3 + \eps^2\|A\|_2^4/3}}
\]
By using that $\|A\|_F \ge \|A\|_2$ and that $\eps \le 1$ and denoting the numeric rank of $A$ by $r = \|A\|_F^2/\|A\|_2^2$ we can simplify this to be
\[
\Pr[\|BB^T - AA^T\| > \eps^2\|A\|_2^2/2] \le 2me^{-\frac{s\eps^4}{16r}}
\]
To conclude, it is enough to sample 
\[
s \ge \frac{16r}{\eps^4}\log(2m/\delta)
\] 
columns from $A$ (with probability proportional to their squared 2 norm) to form a matrix $B$ such that $\|BB^T - AA^T\| \le \eps^2\|A\|_2^2/2$ with probability at least $1-\delta$. According to above this also gives us that $\|A - P_k^BA\| \le \sigma_{k+1} + \eps\|A\|$ which completes our claim.
Note that the number of non zeros in $B$ is bounded by $s\cdot m$ which is independent of $n$, the number of columns.
This is potentially significantly better than the results obtained in last week's class.
 
 
\begin{remark}
More elaborate column selection algorithms exist which provide better approximation but these will not be discussed here.
See \cite{Boutsidis2011} for the latest result that I am aware of.
\end{remark}

\subsection*{Deterministic Lossy SVD}

\newcommand{\diag}{\operatorname{diag}}
\newcommand{\tr}{\operatorname{tr}}

I this section we will see that the above approximation can be improved using a simple trick \cite{LibertyMatrixSketching2012}.
The algorithm keeps an $m \times s$ sketch matrix $B$ which is updated
every time a new column from the input matrix $A$ is added. 
It maintains the invariant that the last column in the sketch is always zero.
When a new input row is added it is places in the last (all zeros) column of the sketch.
Then, using its $SVD$ the sketch is rotated from the right so that its columns are orthogonal.
Finally, the sketch column norms are `shrunk' so that the last column is again all zeros. 
In the algorithm we denote by $[U,\Sigma,V] \leftarrow SVD(B)$ the Singular Value Decomposition of $B$.
We use the convention that $U \Sigma V^T = B$, $U^{T}U = V^{T}V = I$, 
and $\Sigma = \diag([\sigma_1,\ldots,\sigma_s])$,  \;$\sigma_1 \ge \ldots \ge \sigma_s$.
The notation $I$ stands for the $s \times s$ identity matrix while $B_s$ denotes the $s$'th column of $B$ (similarly $A_j$).
\begin{algorithm} 
%\caption{\FD}
\label{alg}
\begin{algorithmic}
\STATE {\bf Input:} $s, \;\;A \in \R^{m \times n}$ 
\STATE $B^{0} \leftarrow $ all zeros matrix $\in \R^{m \times s}$  
\FOR{$i \in [n]$}
	\STATE $C^{i} = B^{i-1}$ \hfill %\# Only for proof notation
	\STATE $C^{i}_{s} = A_i$ \hfill Put $A_i$ in the last (all zeros) column of $C^{i}$
	\STATE $[U^{i},\Sigma^{i},V^{i}] \leftarrow SVD(C^{i})$ %\hfill \# Only for proof notation
	%\STATE \hfill \# $U \Sigma V = B$,  \;$U^{T}U = VV^{T} = I_{\ell}$
	%\STATE \hfill \# $\Sigma = \diag([\sigma_1,\ldots,\sigma_\ell])$,  \;$\sigma_1 \ge \ldots \ge \sigma_\ell$
	\STATE $D^{i} \leftarrow  U^{i}\Sigma^{i}$  %\hfill \# Only for proof notation
	\STATE $\delta_i \leftarrow \Sigma^{i}_{s,s}$ %\hfill \# $\Sigma_{\ell,\ell}$ the least singular value of $B$
	\STATE $W^{i} \leftarrow \sqrt{I - \Sigma^{-2}\delta^2_i}$
	%\STATE \hfill \# $I_{\ell}$ is an $\ell \times \ell$ identity matrix
	\STATE $B^{i} \leftarrow D^{i}W^{i}$ %\hfill \# Here $B_{\ell}$ contains zeros since $\bar{\Sigma}_{\ell,\ell} = 0$
\ENDFOR
\STATE {\bf Return:} $B \leftarrow B^{n}$ 
\end{algorithmic}
\end{algorithm}

\begin{lemma}
Let $B$ be the output of the above algorithm for a matrix $A$ and an integer $s$ then:
\[
\|AA^T - BB^T\| \le \|A\|_F^2/s
\]
\end{lemma}
\begin{proof}
We start by bounding $\|AA^T - BB^T\|$ from above:
\[
\|AA^T - BB^T\| = \max_{\|x\|=1} (x^TAA^Tx - x^TBB^Tx) = \max_{\|x\|=1} (\|x^TA\|^2 - \|x^TB\|^2).
\]
We open this with a simple telescopic sum $\|x^TB\|^2 = \sum_{j=1}^{n}\|x^TB^{i}\|^2 - \|x^TB^{i-1}\|^2$ and by replacing $\|x^TA\|^2 = \sum_{j=1}^{n}(x^TA_j)^2$.
\begin{eqnarray}
\|x^TA\|^2 - \|x^TB\|^2 &=& \sum_{j=1}^{n} [ (x^TA_j)^2  - (\|x^TB^{i}\|^2 - \|x^TB^{i-1}\|^2)]\\
&=& \sum_{j=1}^{n} [ ((x^TA_j)^2  + \|x^TB^{i-1}\|^2)   - \|x^TB^{i}\|^2 ]\\
&=& \sum_{j=1}^{n} [ \|x^TC^{i}\|^2   - \|x^TB^{i}\|^2 ]\\
&=& \sum_{j=1}^{n} [ \|x^TD^{i}\|^2   - \|x^TD^{i}W^{i}\|^2 ]\\
&=& \sum_{j=1}^{n}  x^T[D^{i}(D^{i})^{T}   - D^{i}(W^{i})^{2}(D^{i})^{T}]x \\
&=& \sum_{j=1}^{n}  x^T[\delta_{i}^2 D^{i} (\Sigma^{i})^{-2}(D^{i})^T]x \\
&=& \sum_{j=1}^{n}  x^T[\delta_{i}^2 U^{i}(U^{i})^T]x \\
&\le& \sum_{j=1}^{n} \delta_{i}^2 \|U^{i}(U^{i})^T\| \le \sum_{j=1}^{n} \delta_{i}^2
\end{eqnarray}
%
\noindent For this to mean anything we have to bound the term $\sum_{j=1}^{n} \delta_{i}^2$. We do this computing the Frobenius norm of $B$.
\begin{eqnarray}
\|B^{n}\|_{F}^{2} &=& \sum_{i=1}^{n} \|B^{i}\|_{F}^{2} - \|B^{i-1}\|_{F}^{2}\\
&=& \sum_{i=1}^{n} [\|B^{i}\|_{F}^{2} - \|D^{i}\|_{F}^{2}] + [\|D^{i}\|_{F}^{2} - \|C^{i}\|_{F}^{2}] + [\|C^{i}\|_{F}^{2} -  \|B^{i-1}\|_{F}^{2}]
\end{eqnarray}
Let us deal with each term separately: 
\begin{eqnarray}
\|B^{i}\|_{F}^{2} - \|D^{i}\|_{F}^{2} &=& \tr(B^{i}(B^{i})^{T} - D^{i}(D^{i})^{T}) = \tr(\delta^2_i U^{i}(U^{i})^2) = s\delta^2_i\\
\|D^{i}\|_{F}^{2} - \|C^{i}\|_{F}^{2} &=& 0\\
\|C^{i}\|_{F}^{2} -  \|B^{i-1}\|_{F}^{2} &=& \|A_i\|^2
\end{eqnarray}
Putting these together we get 
\[
\|B\|_{F}^{2} = \|B^{n}\|_{F}^{2}  = \|A\|_F^2 - s \sum \delta_{i}^2
\]
Since $\|B\|_{F}^{2} \ge 0$ we conclude that $\sum \delta_{i}^2 \le \|A\|_F^2/s$
Combining with the above:
\[
\|BB^T - AA^T\| \le \|A\|_F^2/s
\]
\end{proof}
\noindent Finally, we recall that to achieve the approximation guarantee
$\|A - P_k^BA\| \le \sigma_{k+1} + \eps\|A\|$ it is sufficient to require $\|BB^T - AA^T\| \le \eps^2\|A\|_2^2/2$ which is obtained when
\[
s \ge \frac{2r}{\eps^2}
\]
This completes the claim.
\subsection*{Discussion}
Note that while the deterministic lossy SVD procedure requires less space it also requires more operations per column insertion.
Namely, it computes the SVD of the sketch in every iteration. 
This can be somewhat improved (see \cite{LibertyMatrixSketching2012}) but it is still far from being as efficient as column sampling.
Making this algorithm faster while maintaining its approximation guaranty is an open problem.
Another interesting problem is to make this algorithm take advantage of the matrix sparsity.




\bibliographystyle{unsrt}
\bibliography{algorithms_in_data_mining.bib}

\end{document}













%%%%%%%%
