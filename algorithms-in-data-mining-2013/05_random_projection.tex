
\documentclass{article}
\usepackage{fullpage}

%\setlength{\oddsidemargin}{0.25 in}
%\setlength{\evensidemargin}{-0.25 in}
%\setlength{\topmargin}{-0.6 in}
%\setlength{\textwidth}{6.5 in}
%\setlength{\textheight}{8.5 in}
%\setlength{\headsep}{0.75 in}
%\setlength{\parindent}{0 in}
%\setlength{\parskip}{0.1 in}



%%%% PACKAGES %%%%
\usepackage{algorithm,algorithmic}
\usepackage{amsfonts, amsmath, amsthm}
\usepackage{graphicx}
%for dotted lines inside matrices
\usepackage{arydshln}
\setlength{\dashlinedash}{.8pt} %
\setlength{\dashlinegap}{1.2pt}
%

%%%% ENVIRONMENTS %%%%
\newtheorem{definition}{Definition}[section]
\newtheorem{fact}{Fact}[section]
\newtheorem{claim}{Claim}[section]
\newtheorem{lemma}{Lemma}[section]
\newtheorem{remark}{Remark}[section]
\newtheorem{theorem}{Theorem}[section]
\newtheorem{proposition}{Proposition}[section]

%%%% COMMANDS %%%%
\newcommand{\E}{{\mathbb E}}
\newcommand{\Var}{{\operatorname{Var}}}
\newcommand{\var}{{\operatorname{Var}}}
\newcommand{\poly}{{\operatorname{poly}}}
\newcommand{\const}{{\operatorname{const}}}

\newcommand{\allones}{\mathbf{1}}
\newcommand{\abs}[1]{\left| #1 \right|}
\newcommand{\norm}[1]{\| #1 \|}
\newcommand{\eps}{\varepsilon}
\newcommand{\tab}{\hspace{.5cm}}
\newcommand{\R}{{\mathbb{R}}}
\newcommand{\Sph}{{\mathbb{S}}}
\newcommand{\N}{{\mathcal{N}}}


\newcommand{\lecture}[3]{
   %\pagestyle{myheadings}
   %\thispagestyle{plain}
   %\newpage
   %\setcounter{lecnum}{#1}
   %\setcounter{page}{1}
   \noindent
   \begin{center}
   \framebox{
      \vbox{\vspace{2mm}
    \hbox to 6.28in { {\bf 0368-3248-01-Algorithms in Data Mining
		\hfill Fall 2013} }
       \vspace{4mm}
       \hbox to 6.28in { {\Large \hfill Lecture #1: #2  \hfill} }
       \vspace{2mm}
       \hbox to 6.28in { {\it Lecturer: #3 \hfill}}% \hfill Scribes: #4} }
      \vspace{2mm}}
   }
   \end{center}
   \markboth{Lecture #1: #2}{Lecture #1: #2}
{\small
   {\bf Warning}: {\it This note may contain typos and other inaccuracies which are usually discussed during class. 
   Please do not cite this note as a reliable source. If you find mistakes, please inform me.}
   \vspace*{4mm}}
   \hrule
   \vspace{1cm}
}

%%%% COMMON DEFS %%%%
%\author{Edo Liberty \\ Class notes: Algorithms in Data mining}
\date{\nonumber}


\begin{document} %
\lecture{5}{Random-projection}{Edo Liberty}

We will give a simple proof of the following, rather amazing, fact. Every set of $n$ points 
in a Euclidian space (say in dimension $d$) can be embedded into the Euclidian space of 
dimension $k = O(\log(n)/\eps^2)$ such that all pairwise distances are preserved up distortion $1\pm \eps$.
We will prove the construction of \cite{DasGuptaGupta99} which is simpler than the one in \cite{JL84}.

\section*{Random projection}
We will argue that a certain distribution over the choice of a matrix $\R \in \R^{k \times d}$ gives that:
\begin{equation}
\label{e1}
\forall x \in \Sph^{d-1} \;\; \Pr\left[ \left| ||\frac{1}{\sqrt{k}}Rx|| - 1 \right| > \eps \right] \le \frac{1}{n^2} 
\end{equation}
Before we pick this distribution and show that Equation~\ref{e1} holds for it, let us first see
that this gives the opening statement. 

Consider a set of $n$ points $x_1,\ldots, x_n$ in Euclidian space $\R^d$. Embedding these points
into a lower dimension while preserving all distances between
them up to distortion $1\pm \eps$ means approximately preserving the norms of all 
${n \choose 2}$ vectors $x_i - x_j$. Assuming Equation~\ref{e1} holds and using the union 
bound, this property will fail to hold for at least one $x_i - x_j$ pair with probability at most ${n \choose 2}\frac{1}{n^2} \le 1/2$.
Which means that all ${n \choose 2}$ point distances are preserved up to distortion $\eps$ with probability at least $1/2$.


\section{Matrices with normally distributed independent entries}
We consider the distribution of matrices $R$ such that each $R(i,j)$ is drawn independently from  a
normal distribution with mean zero and variance $1$, $R(i,j) \sim \N(0,1)$. We show that for this
distribution Equation~\ref{e1} holds for some $k \in O(\log(n)/\eps^2)$.

First consider the random variable $z = \sum_{j=1}^{d}r(j)x(j)$ where $r(j) \sim \N(0,1)$. 
To understand how the variable $z$ distributes we recall the two-stability of the
normal distribution. Namely, if $z_3 = z_2 + z_1$ and 
$z_1 \sim \N(\mu_1,\sigma_{1})$ and $z_2 \sim \N(\mu_2,\sigma_{2})$ then, $$z_3 \sim \N(\mu_1 + \mu_2,\sqrt{\sigma^{2}_{1} + \sigma^{2}_{2}}).$$
In our case,  $r(i)x(i) \sim \N(0,x_{i})$ and therefore, $z = \sum_{i=1}^{d}r(i)x(i) \sim \N(0,\sqrt{\sum_{i=1}^{d}x^{2}_{i}}) \sim \N(0,1)$.
%
Now, note that each element in the vector $Rx$ distributes exactly like $z$.
Defining $k$ identical copies of $z$, $z_1,\ldots,z_k$,
We get that $||\frac{1}{\sqrt{k}}Rx||$ distributes exactly like $\sqrt{\frac{1}{k}\sum_{i=1}^{k} z^{2}_{i}}$.
Thus, proving Equation~\ref{e1} reduces to showing that:
\begin{equation}
\Pr\left[ \left| \sqrt{\frac{1}{k}\sum_{i=1}^{k} z^{2}_{i}} - 1 \right| > \eps \right] \le \frac{1}{n^2} 
\end{equation}
for a set of independent normal random variables $z_1,\ldots,z_k \sim \N(0,1)$.
It is sufficient to demanding that $\Pr[\sum_{i=1}^{k} z^{2}_{i} \ge k(1+\eps)^2]$ and $\Pr[\sum_{i=1}^{k} z^{2}_{i} \le k(1-\eps)^2]$ are both smaller than $1/2n^2$.
We start with bounding the probability that $\sum_{i=1}^{k} z^{2}_{i} \ge k(1+\eps)$ (this is okay because $k(1+\eps) < k(1+\eps)^2$).
\[
\Pr[\sum z^{2}_{i} \ge k(1+\eps)] = \Pr[e^{\lambda \sum z^{2}_{i}} \le e^{\lambda k (1+\eps)}] \le (\E[e^{\lambda z^2}])^k/e^{\lambda k (1+\eps)}
\]
Since $z \sim \N(0,1)$ we can compute $\E[e^{\lambda z^2}]$ exactly:
\[
\E [e^{\lambda z^{2}}] = \frac{1}{\sqrt{2\pi}}\int_{-\infty}^{\infty} e^{\lambda t^{2}} e^{-\frac{t^{2}}{2}} dt =\frac{1}{\sqrt{2\pi}}\int_{-\infty}^{\infty} e^{-\frac{(t\sqrt{1-2\lambda})^{2}}{2}}dt = e^{\frac{1}{2} \log(1-2\lambda)}
\]
The final step is by substituting $t' = t\sqrt{1-2\lambda}$ and recalling that $\frac{1}{\sqrt{2\pi}}\int_{-\infty}^{\infty} e^{-\frac{t'^{2}}{2}}dt' = 1$.
Finally, using the fact that $log(\frac{1}{1-2\lambda}) \le 2\lambda + 4\lambda^2$ for $\lambda \in [0,1/4]$ we have:
\[
\E [e^{\lambda z^{2}}] = \frac{1}{\sqrt{1-2\lambda}} = e^{\frac{1}{2} \log(\frac{1}{1-2\lambda})} \le e^{\lambda + 2\lambda^2}
\]
Substituting this into the equation above we have that:
\[
\Pr \le e^{k(\lambda  + 2\lambda^2) - k\lambda (1+\eps)} = e^{ 2k\lambda^2 - k\lambda\eps}  = e^{ - k\eps^2/8}  
\]
for $\lambda \leftarrow \eps/4$. Finally, our condition that 
\[
\Pr[\sum_{i=1}^{k} z^{2}_{i} \ge k(1+\eps)] \le e^{ - k\eps^2/8} \le 1/2n^2
\]
is achieved by $k = c\log(n)/\eps^2$.
Calculating for $\Pr[\sum_{i=1}^{k} z^{2}_{i} \le k(1-\eps)]$ in the same manner shows that $k = c\log(n)/\eps^2$ is also sufficient for this case.
This completes the proof.


\bibliographystyle{unsrt}
\bibliography{AlgorithmsInDataMining.bib}

\end{document}














%%%%%%%%
