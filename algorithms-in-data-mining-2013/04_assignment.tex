\documentclass{article}
\usepackage{AlgorithmsInDataMining}
\begin{document}

\lecture{4}{Home Assignment, Due Dec 3rd}{Edo Liberty}

\section{Probabilistic inequalities}
\subsection*{setup}
In this question you will be asked to derive the three most used
probabilistic inequalities for a specific random variable. Let
$x_1,\ldots,x_n$ be independent $\{-1,1\}$ valued random variables.
Each $x_i$ takes the value $1$ with probability $1/2$ and $-1$ else.
Let $X = \sum_{i=1}^{n}x_i$.

\subsection*{questions}
\begin{enumerate}
\item Let the random variable $Y$ be defined as $Y = |X|$.
Prove that Markov's inequality holds for $Y$. Hint: note that $Y$
takes integer values. Also, there is no need to compute $\Pr[Y =
i]$.
\item Prove Chebyshev's inequality for the above random variable
$X$. You can use the fact that Markov's inequality holds for any
positive variable regardless of your success (or lack of if) in the
previous question. Hint: $\var[X] = E[(X-E[X])^2]$.
\item Argue that
\[
\Pr[X > a] = \Pr[\Pi_{i=1}^{n}e^{\lambda x_i} > e^{\lambda a}] \le
\frac{E[\Pi_{i=1}^{n}e^{\lambda x_i}]}{e^{\lambda a}}
\]
for any $\lambda \in [0,1]$. Explain each transition.
\item Argue that:
\[
\frac{E[\Pi_{i=1}^{n}e^{\lambda x_i}]}{e^{\lambda a}} =
\frac{\Pi_{i=1}^{n}E[e^{\lambda x_i}]}{e^{\lambda a}} =
\frac{(E[e^{\lambda x_1}])^n}{e^{\lambda a}}
\]
What properties of the random variables $x_i$ did you use in each
transition?
\item Conclude that $\Pr[X > a] \le e^{-\frac{a^2}{2n}}$ by
showing that:
\[
\exists \;\;\lambda\in [0,1] \;\;s.t. \;\; \frac{(E[e^{\lambda
x_1}])^n}{e^{\lambda a}} \le e^{-\frac{a^2}{2n}}
\]
Hint: For the hyperbolic cosine function we have $\cosh(x) =
\frac{1}{2}(e^{x} + e^{-x}) \le e^{x^2/2}$ for $x \in [0,1]$.
\end{enumerate}
\pagebreak




\section{Approximating the size of a graph}

\subsection*{setup}

In this question we will try to approximate the size of a graph. A
graph $G(V,E)$ is a set of nodes $|V| = n$ and a set of edges $|E| =
m$. Each edge $e \in V\times V$ is a set of two nodes which support
it. We assume the graph is simple which means there are no duplicate
edges and no self loops (i.e. an edge $e=(u,u)$). The degree of a
node, $\deg(u)$, is the number of edges which it supports. More formally
$\deg(u) = |\{e \in E | u \in e\}|$. The degree of each node in the
graph is at least $1$. The question refers to the following sampling
procedure:
\begin{enumerate}
\item $e = (u,v) \leftarrow$ an edge uniformly at random from $E$.
\item with probability $1/2$
\item \tab return $u$
\item else
\item \tab return $v$
\end{enumerate}
Throughout this question we assume that $i)$ we can sample edges
uniformly from the graph $ii)$ that the number of edges $m$ in known $iii)$
that given a node $u$ we can easily compute  $\deg(u)$. The value of $n$,
however, is unknown.
\subsection*{questions}

\begin{enumerate}
\item Let $p(u)$ denote the probability that the sampling procedure returns a specific
node, $u$. Compute $p(u)$ as a function of $\deg(u)$ and $m$. (Note:
$\sum_{u \in V} \deg(u) = 2m$)
\item Let $f(u) = \frac{2m}{\deg(u)}$. Compute:
\[
E_{x \sim smp} [f(x)]
\]
where $x \sim smp$ denotes that $x$ is chosen according to the
distribution on the nodes generated by the above sampling procedure.
\item We say that a graph is $d$-degree-bounded if $\max_{u \in V} \deg(u) \le
d$. Show that for a $d$-degree-bounded graph:
\[
\var_{x \sim smp} [f(x)] \le dn^2
\]
\item Let $Y = \frac{1}{s}\sum_{i=1}^{s} f(x_i)$ where $x_i$ are
nodes chosen independently from the graph according to the above
sampling procedure. Compute $E[Y]$ {\bf and} show that $\var[Y] \le
dn^2/s$.
\item Use Chebyshev's inequality to find a value for $s$ such that 
for any $d$-degree-bounded graph and any two constants $\eps \in
[0,1]$ and $\delta \in [0,1]$:
\[
\Pr[|Y - n| > \eps n ] < \delta.
\]
$s$ should be a function of $d$, $\eps$ and $\delta$.

\end{enumerate}
\pagebreak






\section{Approximate median}
\subsection*{setup}
Given a list $A$ of $n$ numbers $a_1,\ldots,a_n$, we define the rank
of an element $r(a_i)$ as the number of elements which are smaller
than it. For example, the smallest number has rank zero and the
largest has rank $n-1$. Equal elements are ordered arbitrarily. The
median of $A$ is an element $a$ such that $r(a) = n/2$ (rounded
either up or down). An $\alpha$-approximate-median is a number $a$
such that:
\[
n(1/2 - \alpha) \le r(a) \le n(1/2 + \alpha)
\]
In this question we sample $k$ elements uniformly at random {\it
with replacement} from the list $A$. Let the samples be
$\{x_1,\ldots,x_k\} = X$. You will be asked to show that the median of
$X$ is an $\alpha$-approximate-median of $A$.

\subsection*{questions}
\begin{enumerate}
\item What is the probability the a randomly chosen element $x$ is
such that:
\[
r(x) > n(1/2 + \alpha)
\]
\item Let us define $X_{>\alpha}$ as the set of samples whose rank
is greater than $n(1/2 + \alpha)$. More precisely, $X_{>\alpha} =
\{x_i \in X | r(x_i) > n(1/2 + \alpha)\}$. Similarly we define
$X_{<\alpha} = \{x_i \in X | r(x_i) < n(1/2 - \alpha)\}$. Prove that
if $|X_{>\alpha}| < k/2$ and $|X_{<\alpha}| < k/2$ then the median
of $X$ is an $\alpha$-approximate-median of $A$.
\item Let $Z = |X_{>\alpha}|$. Find $t$ for which:
\[
\Pr[Z \ge k/2] = \Pr[Z \ge (1+t)E[Z]]
\]
\item Bound from above the probability that $Z \ge k/2$ as tightly
as possible. If you do so using a probabilistic inequality, justify
your choice.
\item Compute the minimal value for $k$ which will guarantee that
$|X_{>\alpha}| < k/2$ {\bf and} $|X_{<\alpha}| < k/2$ with
probability at least $1-\delta$.
\end{enumerate}
\pagebreak


\section{Simple high capacity hashing}
\subsection*{setup}
In this question we try to evaluate the capacity of a special hash table.
For simplicity, we assume that the hashed elements are a subset of $[N]$ ($[N]$ denots the set $\{1,\dots,N\}$).
The hash table consists of an array $A$ of length $n$ and $L$ perfect hash functions $h_\ell: [N] \rightarrow [n]$.
Throughout the exercise we assume the existence of perfect hash functions. That is, $\Pr[h(x) = i] = 1/n$ for all $x \in [N]$ and $i\in [n]$ 
independently of the values $h(x')$.  For convenience we also assume that the entries in $A$ are initialized to the value $0$.
%
\begin{algorithm}
\caption{$Add(x)$}
\begin{algorithmic}
\FOR {$\ell \in [L]$}
    \IF {$A[h_\ell(x)] == 0$ or $A[h_\ell(x)] == x$}
    	\STATE $A[h_\ell(x)] = x$
	\STATE Return Success
    \ENDIF
\ENDFOR
\STATE Return Fail
\end{algorithmic}
\end{algorithm}
%
\vspace{-.6cm}
\begin{algorithm}
\caption{$Query(x)$}
\begin{algorithmic}
\FOR {$\ell \in [L]$}
    \IF {$A[h_\ell(x)] == x$}
	\STATE Return True
   \ELSIF {$A[h_\ell(x)] == 0$}
   	\STATE Return False
    \ENDIF
\ENDFOR
\STATE Return False
\end{algorithmic}
\end{algorithm}
%
\vspace{-.6cm}
\subsection*{questions}
\begin{enumerate}
\item Argue the correctness of the hashing scheme. 
a) If an element was {\bf successfully} added to the table by $Add(x)$ it will be found by $Query(x)$. 
b) If an element was not added to the table by $Add(x)$ it will not be found by $Query(x)$. 
\item Assume that exactly $m$ cells in the array are occupied. That is, $m$ cells contain values $A[j] > 0$ and for the rest $A[j]=0$.
Given a new element $x$ which is in not stored in the hash table. What is the probability that location $h_1(x)$ in $A$ is occupied.
\item What is the probability that procedure $Add(x)$ fails for an element $x$ not in the hash table? (here we still assume there are exactly $m$ elements already in the table)
\item Assume we start with an empty hash table and insert $m$ elements one after the other. 
Use the union bound to get a value for $L$ for which $Add(x)$ succeeds in {\bf all} $m$ element insertions with probability at least $1-\delta$
\item Argue that the {\bf expected} running time of both $Add(x)$ and $Query(x)$ is $O(1)$. That is, it does not depend on $L$. 
\end{enumerate}



\end{document}
