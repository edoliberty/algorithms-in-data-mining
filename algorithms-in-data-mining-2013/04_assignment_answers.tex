\documentclass{article}
\usepackage{algorithms_in_data_mining}
\begin{document}

\lecture{4}{Home Assignment, Due Dec 3rd}{Edo Liberty}

\section{Probabilistic inequalities}
\subsection*{setup}
In this question you will be asked to derive the three most used
probabilistic inequalities for a specific random variable. Let
$x_1,\ldots,x_n$ be independent $\{-1,1\}$ valued random variables.
Each $x_i$ takes the value $1$ with probability $1/2$ and $-1$ else.
Let $X = \sum_{i=1}^{n}x_i$.

\subsection*{questions}
\begin{enumerate}
\item Let the random variable $Y$ be defined as $Y = |X|$.
Prove that Markov's inequality holds for $Y$. Hint: note that $Y$
takes integer values. Also, there is no need to compute $\Pr[Y =
i]$.
\item Prove Chebyshev's inequality for the above random variable
$X$. You can use the fact that Markov's inequality holds for any
positive variable regardless of your success (or lack of if) in the
previous question. Hint: $\var[X] = E[(X-E[X])^2]$.
\item Argue that
\[
\Pr[X > a] = \Pr[\Pi_{i=1}^{n}e^{\lambda x_i} > e^{\lambda a}] \le
\frac{E[\Pi_{i=1}^{n}e^{\lambda x_i}]}{e^{\lambda a}}
\]
for any $\lambda \in [0,1]$. Explain each transition.
\item Argue that:
\[
\frac{E[\Pi_{i=1}^{n}e^{\lambda x_i}]}{e^{\lambda a}} =
\frac{\Pi_{i=1}^{n}E[e^{\lambda x_i}]}{e^{\lambda a}} =
\frac{(E[e^{\lambda x_1}])^n}{e^{\lambda a}}
\]
What properties of the random variables $x_i$ did you use in each
transition?
\item Conclude that $\Pr[X > a] \le e^{-\frac{a^2}{2n}}$ by
showing that:
\[
\exists \;\;\lambda\in [0,1] \;\;s.t. \;\; \frac{(E[e^{\lambda
x_1}])^n}{e^{\lambda a}} \le e^{-\frac{a^2}{2n}}
\]
Hint: For the hyperbolic cosine function we have $\cosh(x) =
\frac{1}{2}(e^{x} + e^{-x}) \le e^{x^2/2}$ for $x \in [0,1]$.
\end{enumerate}
\pagebreak


%%%%%%%%%%%%%%%%%%%%%%%%%%%%%%%%%%%%%%%%%%%%%%%
%%%%%%%%%%%%%%%%%%%%%%%%%%%%%%%%%%%%%%%%%%%%%%%
%%%%%%%%%%%%%%%%%%%%%%%%%%%%%%%%%%%%%%%%%%%%%%%


\subsection*{answers}
\begin{enumerate}
\item 
\begin{eqnarray*}
E[Y] &=& \sum_{i=0}^{n}\Pr[Y=i]\cdot i\\
&=& \sum_{i=0}^{t}\Pr[Y=i]\cdot i + \sum_{i=t+1}^{n}\Pr[Y=i]\cdot i\\
&\ge& \sum_{i=t+1}^{n}\Pr[Y=i]\cdot i \\
&\ge& \sum_{i=t+1}^{n}\Pr[Y=i]\cdot t \\
&=& t\cdot\Pr[Y > t]
\end{eqnarray*}
Therefore, $E[Y] \ge t\cdot\Pr[Y > t]$ which is Markov's inequality.
\item This is identical to the general proof of Chebyshev's inequality.
We define $Z = (X - E[X])^2$. Since $Z$ is positive we can use Markov's inequality for it and get:
\[
\Pr[|X - E[X]| > t] = \Pr[Z > t^2] \le \frac{E[Z]}{t^2} = \frac{\var[X]}{t^2} 
\]
Here we used that $E[Z] = E[(X - E[X])^2] = \var[X]$.
\item First transition:
\[
\Pr[X > a] = \Pr[\lambda X > \lambda a] = \Pr[e^{\lambda X} > e^{\lambda a}] = \Pr[e^{\lambda \sum x_i} > e^{\lambda a}] = \Pr[\Pi_{i=1}^{n}e^{\lambda x_i} > e^{\lambda a}]
\]
These hold due to the monotonicity of multiplication by a positive constant and exponentiation.
Now, using Markov's inequality on the last inequality we get:
\[
\Pr[\Pi_{i=1}^{n}e^{\lambda x_i} > e^{\lambda a}] \le \frac{E[\Pi_{i=1}^{n}e^{\lambda x_i}]}{e^{\lambda a}} 
\]
\item The first transition is true due to the independence of the variables $x_i$. This means that $e^{\lambda x_i}$ are independent.
The second transition is due to all expectations of $e^{\lambda x_i}$ being equal which stems from $x_i$ being identically distributed.
\item First, we compute the expectation of $e^{\lambda x_i}$ 
\[
E[e^{\lambda x_i}] = \frac{1}{2}e^{\lambda} + \frac{1}{2}e^{-\lambda} = \cosh(\lambda) \le e^{\lambda^2/2}
\]
From the above we have that $\Pr[X > a] \le e^{n\lambda^2/2  - \lambda a}$. Setting $\lambda = a/n$ we get 
$e^{n\lambda^2/2  - \lambda a} = e^{-\frac{a^2}{2n}}$ which concludes the proof.
\end{enumerate}
\pagebreak

%%%%%%%%%%%%%%%%%%%%%%%%%%%%%%%%%%%%%%%%%%%%%%%%%%%%%%%
%%%%%%%%%%%%%%%%%%%%%%%%%%%%%%%%%%%%%%%%%%%%%%%%%%%%%%%
%%%%%%%%%%%%%%%%%%%%%%%%%%%%%%%%%%%%%%%%%%%%%%%%%%%%%%%
%%%%%%%%%%%%%%%%%%%%%%%%%%%%%%%%%%%%%%%%%%%%%%%%%%%%%%%


\section{Approximating the size of a graph}

\subsection*{setup}

In this question we will try to approximate the size of a graph. A
graph $G(V,E)$ is a set of nodes $|V| = n$ and a set of edges $|E| =
m$. Each edge $e \in V\times V$ is a set of two nodes which support
it. We assume the graph is simple which means there are no duplicate
edges and no self loops (i.e. an edge $e=(u,u)$). The degree of a
node, $\deg(u)$, is the number of edges which it supports. More formally
$\deg(u) = |\{e \in E | u \in e\}|$. The degree of each node in the
graph is at least $1$. The question refers to the following sampling
procedure:
\begin{enumerate}
\item $e = (u,v) \leftarrow$ an edge uniformly at random from $E$.
\item with probability $1/2$
\item \tab return $u$
\item else
\item \tab return $v$
\end{enumerate}
Throughout this question we assume that $i)$ we can sample edges
uniformly from the graph $ii)$ that the number of edges $m$ in known $iii)$
that given a node $u$ we can easily compute  $\deg(u)$. The value of $n$,
however, is unknown.
\subsection*{questions}

\begin{enumerate}
\item Let $p(u)$ denote the probability that the sampling procedure returns a specific
node, $u$. Compute $p(u)$ as a function of $\deg(u)$ and $m$. (Note:
$\sum_{u \in V} \deg(u) = 2m$)
\item Let $f(u) = \frac{2m}{\deg(u)}$. Compute:
\[
E_{x \sim smp} [f(x)]
\]
where $x \sim smp$ denotes that $x$ is chosen according to the
distribution on the nodes generated by the above sampling procedure.
\item We say that a graph is $d$-degree-bounded if $\max_{u \in V} \deg(u) \le
d$. Show that for a $d$-degree-bounded graph:
\[
\var_{x \sim smp} [f(x)] \le dn^2
\]
\item Let $Y = \frac{1}{s}\sum_{i=1}^{s} f(x_i)$ where $x_i$ are
nodes chosen independently from the graph according to the above
sampling procedure. Compute $E[Y]$ {\bf and} show that $\var[Y] \le
dn^2/s$.
\item Use Chebyshev's inequality to find a value for $s$ such that 
for any $d$-degree-bounded graph and any two constants $\eps \in
[0,1]$ and $\delta \in [0,1]$:
\[
\Pr[|Y - n| > \eps n ] < \delta.
\]
$s$ should be a function of $d$, $\eps$ and $\delta$.

\end{enumerate}
\pagebreak

%%%%%%%%%%%%%%%%%%%%%%%%%%%%%%%%%%%%%%%%%%%%%%%%%%%%%%%
%%%%%%%%%%%%%%%%%%%%%%%%%%%%%%%%%%%%%%%%%%%%%%%%%%%%%%%
%%%%%%%%%%%%%%%%%%%%%%%%%%%%%%%%%%%%%%%%%%%%%%%%%%%%%%%
%%%%%%%%%%%%%%%%%%%%%%%%%%%%%%%%%%%%%%%%%%%%%%%%%%%%%%%


\subsection*{answers}
\begin{enumerate}
\item A node is chosen only if an edge it is adjacent to is picked with probability and then it is the node picked between the two.
The first event happens with probability $deg(u)/m$ since the edges a re chosen uniformly at random.
The second event happens with probability $1/2$ independently of the first event. This gives $p(u) = \frac{deg(u)}{m}\frac{deg(2)}{2} = \frac{deg(u)}{2m}$. 


\item By the definition to the expectation:
\[
E_{x \sim smp} [f(x)] = \sum_{u\in V} p(u) f(u) = \sum_{u\in V}  \frac{deg(u)}{2m} \frac{2m}{\deg(u)} =  \sum_{u\in V}1 = n
\]

\item We say that a graph is $d$-degree-bounded if $\max_{u \in V} \deg(u) \le
d$. Show that for a $d$-degree-bounded graph:
\[
\var_{x \sim smp} [f(x)] \le E_{x \sim smp} [f^2(x)]  = \sum_{u\in V}\frac{deg(u)}{2m} (\frac{2m}{\deg(u)})^2 =  \sum_{u\in V}\frac{2m}{\deg(u)} 
\]
Since $deg(u) \ge 1$ then  $\sum_{u\in V}\frac{2m}{\deg(u)} \le  \sum_{u\in V}\frac{2m}{1} = 2mn$.
Also, since the graph is $d$-degree-bounded $2m = \sum_{u \in V}deg(u) \le nd$ thus $ 2mn \le dn^2$.

\item $Y$ is the average of $s$ independent copies of $f(x)$ and therefore, by linearity of the expectation, 
we have that $E[Y] = E[f] = n$. Moreover, Since the nodes $x_i$ are chosen independently we have that 
$\var[Y] = \frac{1}{s^2}\sum_{i=1}^{s}\var[f(x_i)]$. Since $f(x_i)$ distribute identically and 
substituting $\var(x) \le dn^2$ we get $\frac{1}{s^2}\sum_{i=1}^{s}\var[f(x_i)]  \le \frac{s}{s^2}dn^2 = dn^2/s$.

\item Since $E[Y] = n$ we get that the above holds if 
\[
\Pr[|Y - E[n]| > \eps n ] < \frac{\var[Y]}{\eps^2 n^2} \le \frac{dn^2/s}{\eps^2 n^2}  = \frac{d}{s \eps^2}
\]
The condition that $\frac{d}{s \eps^2} \le \delta$ holds for $s \ge \frac{d}{\delta \eps^2}$
\end{enumerate}
\pagebreak



%%%%%%%%%%%%%%%%%%%%%%%%%%%%%%%%%%%%%%%%%%%%%%%%%%%%%%%
%%%%%%%%%%%%%%%%%%%%%%%%%%%%%%%%%%%%%%%%%%%%%%%%%%%%%%%
%%%%%%%%%%%%%%%%%%%%%%%%%%%%%%%%%%%%%%%%%%%%%%%%%%%%%%%
%%%%%%%%%%%%%%%%%%%%%%%%%%%%%%%%%%%%%%%%%%%%%%%%%%%%%%%

\section{Approximate median}
\subsection*{setup}
Given a list $A$ of $n$ numbers $a_1,\ldots,a_n$, we define the rank
of an element $r(a_i)$ as the number of elements which are smaller
than it. For example, the smallest number has rank zero and the
largest has rank $n-1$. Equal elements are ordered arbitrarily. The
median of $A$ is an element $a$ such that $r(a) = n/2$ (rounded
either up or down). An $\alpha$-approximate-median is a number $a$
such that:
\[
n(1/2 - \alpha) \le r(a) \le n(1/2 + \alpha)
\]
In this question we sample $k$ elements uniformly at random {\it
with replacement} from the list $A$. Let the samples be
$\{x_1,\ldots,x_k\} = X$. You will be asked to show that the median of
$X$ is an $\alpha$-approximate-median of $A$.

\subsection*{questions}
\begin{enumerate}
\item What is the probability the a randomly chosen element $x$ is
such that:
\[
r(x) > n(1/2 + \alpha)
\]
\item Let us define $X_{>\alpha}$ as the set of samples whose rank
is greater than $n(1/2 + \alpha)$. More precisely, $X_{>\alpha} =
\{x_i \in X | r(x_i) > n(1/2 + \alpha)\}$. Similarly we define
$X_{<\alpha} = \{x_i \in X | r(x_i) < n(1/2 - \alpha)\}$. Prove that
if $|X_{>\alpha}| < k/2$ and $|X_{<\alpha}| < k/2$ then the median
of $X$ is an $\alpha$-approximate-median of $A$.
\item Let $Z = |X_{>\alpha}|$. Find $t$ for which:
\[
\Pr[Z \ge k/2] = \Pr[Z \ge (1+t)E[Z]]
\]
\item Bound from above the probability that $Z \ge k/2$ as tightly
as possible. If you do so using a probabilistic inequality, justify
your choice.
\item Compute the minimal value for $k$ which will guarantee that
$|X_{>\alpha}| < k/2$ {\bf and} $|X_{<\alpha}| < k/2$ with
probability at least $1-\delta$.
\end{enumerate}
\pagebreak

%%%%%%%%%%%%%%%%%%%%%%%%%%%%%%%%%%%%%%%%%%%%%%%%%%%%%%%
%%%%%%%%%%%%%%%%%%%%%%%%%%%%%%%%%%%%%%%%%%%%%%%%%%%%%%%
%%%%%%%%%%%%%%%%%%%%%%%%%%%%%%%%%%%%%%%%%%%%%%%%%%%%%%%
%%%%%%%%%%%%%%%%%%%%%%%%%%%%%%%%%%%%%%%%%%%%%%%%%%%%%%%

\subsection*{answers}
\begin{enumerate}
\item There are $n(1/2 - \alpha)$ elements for which $r(x) > n(1/2 + \alpha)$.
Since the element is chosen uniformly, the probability of that happening is $(1/2 - \alpha)$.
\item First we note that the median of $X$ cannot be either in $X_{>\alpha}$ or in $X_{<\alpha}$.
This is simply because each of them includes less than half of the elements in $X$.
Moreover, by the definitions of $X_{>\alpha}$ and $X_{<\alpha}$ we have:
\[
n(1/2 - \alpha) \le r(median(X)) \;\;\;\;\mbox{and}\;\;\;\;\;  r(median(X))  \le n(1/2 + \alpha)
\]
which means that $median(X)$ is an $\alpha$-approximate-median of $A$.
\item Since the probability of a sample being in $X_{>\alpha}$ is exactly $1/2-\alpha$ 
and since we have $k$ independent samples, $E[Z] = E[|X_{>\alpha}|] = k(1/2-\alpha)$.
Solving for $t$ we get 
\[
(1+t)E[Z] = k/2 \;\;\rightarrow\;\;  (1+t)(1/2-\alpha) = 1/2 \;\;\;\rightarrow\;\;\; t = \frac{2\alpha}{1-2\alpha}
\]
\item Since the value of $Z$ is the sum of independent indicator variables we can apply Chernoff's inequality.
Denoting $\mu = E[Z] = k(1/2-\alpha)$ and $t = \frac{2\alpha}{1-2\alpha}$ we have:
\[
\Pr[Z \ge k/2] = \Pr[Z \ge (1+t)\mu] \le e^{-\mu t^2/4} 
\] 

\item Similarly to the the above we can argue that 
\[
\Pr[|X_{<\alpha}| \ge k/2] \le e^{-\mu t^2/4}
\]
From the union bound we have that the probability of the event that $|X_{<\alpha}| \ge k/2$
or that $|X_{>\alpha}| \ge k/2$ is at most the sum of their probabilities.
\[
\Pr \left[|X_{<\alpha}| \ge k/2 \cup |X_{>\alpha}| \ge k/2 \right] \le \Pr\left[|X_{<\alpha}| \ge k/2\right] + \Pr[|X_{>\alpha}| \ge k/2] \le 2e^{-\mu t^2/4}
\] 
Demanding that this failure probability is less than $\delta$ we guarantee success with probability at least $1-\delta$.
Substituting $\mu = k(1/2- \alpha)$ and $t = \frac{2\alpha}{1-2\alpha}$ this is achieved for 
\[
2e^{-\mu t^2/4} < \delta \;\;\;\rightarrow\;\;\; k > \frac{4\log(2/\delta)(1/2-\alpha)}{\alpha^2}  
\]
\end{enumerate}

\pagebreak

%%%%%%%%%%%%%%%%%%%%%%%%%%%%%%%%%%%%%%%%%%%%%%%%%%%%%%%
%%%%%%%%%%%%%%%%%%%%%%%%%%%%%%%%%%%%%%%%%%%%%%%%%%%%%%%
%%%%%%%%%%%%%%%%%%%%%%%%%%%%%%%%%%%%%%%%%%%%%%%%%%%%%%%
%%%%%%%%%%%%%%%%%%%%%%%%%%%%%%%%%%%%%%%%%%%%%%%%%%%%%%%





\section{Simple high capacity hashing}
\subsection*{setup}
In this question we try to evaluate the capacity of a special hash table.
For simplicity, we assume that the hashed elements are a subset of $[N]$ ($[N]$ denots the set $\{1,\dots,N\}$).
The hash table consists of an array $A$ of length $n$ and $L$ perfect hash functions $h_\ell: [N] \rightarrow [n]$.
Throughout the exercise we assume the existence of perfect hash functions. That is, $\Pr[h(x) = i] = 1/n$ for all $x \in [N]$ and $i\in [n]$ 
independently of the values $h(x')$.  For convenience we also assume that the entries in $A$ are initialized to the value $0$.
%
\begin{algorithm}
\caption{$Add(x)$}
\begin{algorithmic}
\FOR {$\ell \in [L]$}
    \IF {$A[h_\ell(x)] == 0$ or $A[h_\ell(x)] == x$}
    	\STATE $A[h_\ell(x)] = x$
	\STATE Return Success
    \ENDIF
\ENDFOR
\STATE Return Fail
\end{algorithmic}
\end{algorithm}
%
\vspace{-.6cm}
\begin{algorithm}
\caption{$Query(x)$}
\begin{algorithmic}
\FOR {$\ell \in [L]$}
    \IF {$A[h_\ell(x)] == x$}
	\STATE Return True
   \ELSIF {$A[h_\ell(x)] == 0$}
   	\STATE Return False
    \ENDIF
\ENDFOR
\STATE Return False
\end{algorithmic}
\end{algorithm}
%
\vspace{-.6cm}
\subsection*{questions}
\begin{enumerate}
\item Argue the correctness of the hashing scheme. 
a) If an element was {\bf successfully} added to the table by $Add(x)$ it will be found by $Query(x)$. 
b) If an element was not added to the table by $Add(x)$ it will not be found by $Query(x)$. 
\item Assume that exactly $m$ cells in the array are occupied. That is, $m$ cells contain values $A[j] > 0$ and for the rest $A[j]=0$.
Given a new element $x$ which is in not stored in the hash table. What is the probability that location $h_1(x)$ in $A$ is occupied.
\item What is the probability that procedure $Add(x)$ fails for an element $x$ not in the hash table? (here we still assume there are exactly $m$ elements already in the table)
\item Assume we start with an empty hash table and insert $m$ elements one after the other. 
Use the union bound to get a value for $L$ for which $Add(x)$ succeeds in {\bf all} $m$ element insertions with probability at least $1-\delta$
\item Argue that the {\bf expected} running time of both $Add(x)$ and $Query(x)$ is $O(1)$. That is, it does not depend on $L$. 
\end{enumerate}



%%%%%%%%%%%%%%%%%%%%%%%%%%%%%%%%%%%%%%%%%%%%%%%%%%%%%%%
%%%%%%%%%%%%%%%%%%%%%%%%%%%%%%%%%%%%%%%%%%%%%%%%%%%%%%%
%%%%%%%%%%%%%%%%%%%%%%%%%%%%%%%%%%%%%%%%%%%%%%%%%%%%%%%
%%%%%%%%%%%%%%%%%%%%%%%%%%%%%%%%%%%%%%%%%%%%%%%%%%%%%%%
\pagebreak

\subsection*{answers}
\begin{enumerate}
\item If $Add(x)$ returned ``success" then for some $\ell$ we have $A[h_\ell(x)] = x$ and for any $\ell' < \ell$ it holds that $A[h_{\ell'}(x)] \not\in \{0,x\}$. 
Therefore it will be found by $Query(x)$. Also, if $x$ was not added than it cannot be found by $Query$ since it returns ``True" only if $A[h_\ell(x)] = x$ for some $\ell$. 
\item Since $x$ is was not added and since $h_1$ is a perfect hash function then $\Pr[h_1(x) = i]  = 1/n$ for all $i \in [n]$.
Since there are $m$ occupied cells this sums to $\Pr[A[h_1(x)] > 0] = m/n$.
\item $Add$ fails only if for each to the $\ell \in [L]$ hash functions $A[h_\ell(x)] > 0$. Since they are chosen independently of each other we have 
\[
\Pr[Add(x) \mbox{\; fails}]=(m/n)^L
\]
\item Using the union bound we have that $\Pr[fail] \le \sum_{i \in [m]} ((i-1)/n)^L$. This is because there are at most $i-1$ elements in the hash table when we insert the $i$'th one. Computing this sum can be made simpler by bounding it with an integral.
\[
\sum_{i \in [m]} ((i-1)/n)^L \le \int_{t=1}^{m+1}((t-1)/n)^L dt = \int_{t=0}^{m}(t/n)^L dt = \frac{1}{L+1}(m/n)^{L+1}
\] 
That said, even a bound as simple as $m (m/n)^L$ would have sufficed. For the sake of simplicity let us use the latter.
We obtain that the failure probability is $m (m/n)^L \le \delta$ if $L \ge \log(m/\delta)/\log(n/m)$.
Note that the hash can contain millions of items and be at $\sim 80\%$ capacity and still $L \sim 100$.
\item Let us start with the expected running time of $Add$.  Denote by $\ell^{*} = \min_\ell A[h_\ell(x)] = 0$. 
Clearly, the running is $O(\ell^{*})$ since each lookup requires $O(1)$ time.
\[
\E[\ell^{*}] = \sum_{\ell=1}^{L} \ell  \Pr[\ell^{*} = \ell] \le  \sum_{\ell=1}^{\infty} \ell (\frac{m}{n})^{\ell-1}  (1- \frac{m}{n}) = O(1)
\]
This assumes the ratio between $m$ and $n$ is fixed. Regardless, this does not depend on $L$.

Now we argue the same about $Query$. If $x$ has been added then $Query(x)$ takes the same amount of time that $Add(x)$ did at the time of insertion.
If $x$ has not been added then $Query$ returns $False$ in the same amount of time it would have taken to run $Add(x)$.
If both both cases it reduces the calculation above.





\end{enumerate}

\end{document}
