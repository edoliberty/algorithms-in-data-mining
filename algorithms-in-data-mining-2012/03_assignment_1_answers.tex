\documentclass{article}


\documentclass{article}
\usepackage{fullpage}

%\setlength{\oddsidemargin}{0.25 in}
%\setlength{\evensidemargin}{-0.25 in}
%\setlength{\topmargin}{-0.6 in}
%\setlength{\textwidth}{6.5 in}
%\setlength{\textheight}{8.5 in}
%\setlength{\headsep}{0.75 in}
%\setlength{\parindent}{0 in}
%\setlength{\parskip}{0.1 in}



%%%% PACKAGES %%%%
\usepackage{algorithm,algorithmic}
\usepackage{amsfonts, amsmath, amsthm}
\usepackage{graphicx}
%for dotted lines inside matrices
\usepackage{arydshln}
\setlength{\dashlinedash}{.8pt} %
\setlength{\dashlinegap}{1.2pt}
%

%%%% ENVIRONMENTS %%%%
\newtheorem{definition}{Definition}[section]
\newtheorem{fact}{Fact}[section]
\newtheorem{claim}{Claim}[section]
\newtheorem{lemma}{Lemma}[section]
\newtheorem{remark}{Remark}[section]
\newtheorem{theorem}{Theorem}[section]
\newtheorem{proposition}{Proposition}[section]

%%%% COMMANDS %%%%
\newcommand{\E}{{\mathbb E}}
\newcommand{\Var}{{\operatorname{Var}}}
\newcommand{\var}{{\operatorname{Var}}}
\newcommand{\poly}{{\operatorname{poly}}}
\newcommand{\const}{{\operatorname{const}}}

\newcommand{\allones}{\mathbf{1}}
\newcommand{\abs}[1]{\left| #1 \right|}
\newcommand{\norm}[1]{\| #1 \|}
\newcommand{\eps}{\varepsilon}
\newcommand{\tab}{\hspace{.5cm}}
\newcommand{\R}{{\mathbb{R}}}
\newcommand{\Sph}{{\mathbb{S}}}
\newcommand{\N}{{\mathcal{N}}}


\newcommand{\lecture}[3]{
   %\pagestyle{myheadings}
   %\thispagestyle{plain}
   %\newpage
   %\setcounter{lecnum}{#1}
   %\setcounter{page}{1}
   \noindent
   \begin{center}
   \framebox{
      \vbox{\vspace{2mm}
    \hbox to 6.28in { {\bf 0368-3248-01-Algorithms in Data Mining
		\hfill Fall 2013} }
       \vspace{4mm}
       \hbox to 6.28in { {\Large \hfill Lecture #1: #2  \hfill} }
       \vspace{2mm}
       \hbox to 6.28in { {\it Lecturer: #3 \hfill}}% \hfill Scribes: #4} }
      \vspace{2mm}}
   }
   \end{center}
   \markboth{Lecture #1: #2}{Lecture #1: #2}
{\small
   {\bf Warning}: {\it This note may contain typos and other inaccuracies which are usually discussed during class. 
   Please do not cite this note as a reliable source. If you find mistakes, please inform me.}
   \vspace*{4mm}}
   \hrule
   \vspace{1cm}
}

%%%% COMMON DEFS %%%%
%\author{Edo Liberty \\ Class notes: Algorithms in Data mining}
\date{\nonumber}



\title{Assignment 1} %

\begin{document} %
\date{\nonumber}
\maketitle


\section{Approximating the size of a tree}
\subsection*{setup}
In this question we will try to approximate the number of leaves in
a tree. A binary tree is a graph consisting of internal nodes
and $n$ leaves. Each internal node, $u$, has two children. A left
child $l(u)$ and a right child $r(u)$. The only node which does not have
a parent is the root of the tree $u_{root}$. For each node we also
denote by $d(u)$ its depth in the tree which is the distance from
the root. For example $d(u_{rood}) = 0$ and $d(r(u_{root}))=1$.

We define a random walk on a tree as the process of starting at the
root and then randomly moving to one of the children until we hit a
leaf. More precisely:
\begin{enumerate}
\item $u \leftarrow u_{root}$
\item while $u$ is an internal node
\item $\tab$ w.p. $1/2$
\item \tab \tab $u \leftarrow l(u)$
\item \tab otherwise
\item \tab \tab $u \leftarrow r(u)$
\item return $u$
\end{enumerate}

\subsection*{questions}
\begin{enumerate}
\item Let the leaf $u$ be at depth $d(u)$. Calculate the
probability, $p(u)$, that the random walk outputs $u$?
\item Let $x$ be the output leaf of a random walk and let $f(x) =
2^{d(x)}$ be a function defined on the leaves. Compute the value of:
\[
E_{x\sim w}[f(x)]
\]
where $x\sim w$ denotes that $x$ is chosen according to the
distribution on the leaves generated by the random walk.
\item We say that a tree is $c$-balanced if $d(u) \le \log_2{n} + c$ for all leaves in the tree.
Show that for a $c$-balanced tree
\[
\Var_{x\sim w}[f(x)] \le 2^{c}n^2
\]
\item Let $Y = \frac{1}{s}\sum_{i=1}^{s}f(x_i)$ where $x_i$ are
output nodes of $s$ independent random walks on the tree. Compute
$E[Y]$ {\bf and} show that $\var[Y] \le 2^{c}n^2/s$.
\item Use Chebyshev's inequality to find a value for $s$ such that for two constants
$\eps \in [0,1]$ and $\delta \in [0,1]$:
\[
\Pr[|Y - n| > \eps n ] < \delta.
\]
$s$ should be a function of $c$, $\eps$ and $\delta$.
\end{enumerate}
%\pagebreak
\section{Answers}
\begin{enumerate}
\item
\item
\item
\item
\item
\end{enumerate}

\section{Approximate histograms}
\subsection*{setup}
We are given a stream of elements $x_1,\ldots, x_N$ where $x_i \in \{a_1,\ldots,a_n\}$.
Let $n_i$ denote the number of times element $a_i$ appeared in the stream, i.e., $n_i = |\{ j | x_j = a_i \}|$.
Our goal is to estimate $n_i$ for all frequent elements.
Let the sub stream $y$ include every element in the stream $x$ with probability $p$.
let $\hat{n}_i = |\{ j | y_j = a_i \}|$ be the number of times $a_i$ appears in $y$.  
\subsection*{questions}
\begin{enumerate}
\item Let $z_i = \hat{n}_{i}/p$, compute $\E[z_i]$
\item Assume $a_1$ is such that $n_1 \ge \theta N$ for some fixed $\theta$.
Compute a value for $p$ (as low as possible) which guaranties that $n_1 (1+\eps)\ge z_1 \ge n_1 (1-\eps)$ w.p. at least $1/2$. 
\item Assume $a_1$ is such that $n_1 < \theta N(1-2\eps)$ for some fixed $\theta$.
Compute a value for $p$ (as low as possible) which gives that $z_1 \le \theta N(1-\eps)$ w.p. at least $1/2$. 
\item Use the union bound to specify a value for $p$ which guaranties that for every $i$, if $n_i \ge \theta N$
then $n_i (1+\eps)\ge z_i \ge n_i (1-\eps)$ and if $n_i < \theta N (1-2\eps)$ then $z_i \le \theta N (1-\eps)$ with probability at least $1-\delta$. 
\item Compare this result with the algorithm described in class for approximately counting frequent items in streams, which is better under what circumstances?
\end{enumerate}



\section{Answers}
\begin{enumerate}
\item
\item
\item
\item
\item
\end{enumerate}



\end{document}
