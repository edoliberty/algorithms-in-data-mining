\documentclass{article}

\documentclass{article}
\usepackage{fullpage}

%\setlength{\oddsidemargin}{0.25 in}
%\setlength{\evensidemargin}{-0.25 in}
%\setlength{\topmargin}{-0.6 in}
%\setlength{\textwidth}{6.5 in}
%\setlength{\textheight}{8.5 in}
%\setlength{\headsep}{0.75 in}
%\setlength{\parindent}{0 in}
%\setlength{\parskip}{0.1 in}



%%%% PACKAGES %%%%
\usepackage{algorithm,algorithmic}
\usepackage{amsfonts, amsmath, amsthm}
\usepackage{graphicx}
%for dotted lines inside matrices
\usepackage{arydshln}
\setlength{\dashlinedash}{.8pt} %
\setlength{\dashlinegap}{1.2pt}
%

%%%% ENVIRONMENTS %%%%
\newtheorem{definition}{Definition}[section]
\newtheorem{fact}{Fact}[section]
\newtheorem{claim}{Claim}[section]
\newtheorem{lemma}{Lemma}[section]
\newtheorem{remark}{Remark}[section]
\newtheorem{theorem}{Theorem}[section]
\newtheorem{proposition}{Proposition}[section]

%%%% COMMANDS %%%%
\newcommand{\E}{{\mathbb E}}
\newcommand{\Var}{{\operatorname{Var}}}
\newcommand{\var}{{\operatorname{Var}}}
\newcommand{\poly}{{\operatorname{poly}}}
\newcommand{\const}{{\operatorname{const}}}

\newcommand{\allones}{\mathbf{1}}
\newcommand{\abs}[1]{\left| #1 \right|}
\newcommand{\norm}[1]{\| #1 \|}
\newcommand{\eps}{\varepsilon}
\newcommand{\tab}{\hspace{.5cm}}
\newcommand{\R}{{\mathbb{R}}}
\newcommand{\Sph}{{\mathbb{S}}}
\newcommand{\N}{{\mathcal{N}}}


\newcommand{\lecture}[3]{
   %\pagestyle{myheadings}
   %\thispagestyle{plain}
   %\newpage
   %\setcounter{lecnum}{#1}
   %\setcounter{page}{1}
   \noindent
   \begin{center}
   \framebox{
      \vbox{\vspace{2mm}
    \hbox to 6.28in { {\bf 0368-3248-01-Algorithms in Data Mining
		\hfill Fall 2013} }
       \vspace{4mm}
       \hbox to 6.28in { {\Large \hfill Lecture #1: #2  \hfill} }
       \vspace{2mm}
       \hbox to 6.28in { {\it Lecturer: #3 \hfill}}% \hfill Scribes: #4} }
      \vspace{2mm}}
   }
   \end{center}
   \markboth{Lecture #1: #2}{Lecture #1: #2}
{\small
   {\bf Warning}: {\it This note may contain typos and other inaccuracies which are usually discussed during class. 
   Please do not cite this note as a reliable source. If you find mistakes, please inform me.}
   \vspace*{4mm}}
   \hrule
   \vspace{1cm}
}

%%%% COMMON DEFS %%%%
%\author{Edo Liberty \\ Class notes: Algorithms in Data mining}
\date{\nonumber}


\title{Matrix Sampling}
\date{\nonumber}

\begin{document} %
\maketitle


\section{Ashwelde-Winter inequality}
In their work \cite{AhlswedeW02} Ashwelde and Winter give an incredibly useful bound for 
the sums of independent random symmetric matrices.
Here we recap the lemma statement. 
A short proof due to Roman Vershynin \cite{VershyninMatrixChernoffBounds} is given as a reference.
 
\begin{lemma}
Let $X_i$ be independent random $d \times d$ symmetric matrices with mean zero s.t. $\|X_i\| \le 1$.
Let $S_n = \sum_{i=1}^{n}X_i$, let $\sigma^2_i = \|\Var[X_i]\|$ and $\sigma^2 = \sum_{i=1}^{n}\sigma^2_i$ , then:
\[
\Pr[\| S_n - \E[S_n] \| \ge t] \le d\cdot \max\{e^{-\frac{t^2}{4\sigma^2}}, e^{\frac{-t}{2}}\}
\] 
\end{lemma}

\section{Rank-k approximation}

Here we try to approximate $AA^T$ by sampling columns of the matrix $A$.
From this point onwards we assume, w.l.o.g. that $\|A\|_{fro} = 1$.

Define $n$ unit norm matrices $C_i = A_{(i)}A_{(i)}^T/\|A_{(i)}\|^2$ where $A_{i}$ is the $i$'th column of $A$.
Also define the random matrix valued variable $Z$ which takes values $C_i$ w.p. $p_i = \|A_{(i)}\|^2$.
Note that $p$ is a distribution since $\sum_{i=1}^{n}p_i = \sum_{i=1}^{n}\|A_{(i)}\|^2 = \|A\|_{fro}^2 = 1$.
Let us compute the expectation of $Z$:
\[
\E[Z] = \sum_{i=1}^{n}p_i C_i = \sum_{i=1}^{n}  \|A_{(i)}\|^2 (A_{(i)}A_{(i)}^T/\|A_{(i)}\|^2) =  \sum_{i=1}^{n} A_{(i)}A_{(i)}^T = AA^T
\]
We will therefore try to approximate $AA^T$ by averaging $r$ independent copies of such variables $\frac{1}{r}\sum_{i=1}^{r}Z_i$.
\begin{eqnarray}
\Pr[ \| \frac{1}{r}\sum_{i=1}^{r}Z_i - AA^T \| &>& \eps\|AA^T \|] = \Pr[ \| \sum_{i=1}^{r}(Z_i - AA^T) \| > r\eps\|AA^T \|] \\
&=& \Pr[ \| \sum_{i=1}^{r}X_i \| > r\eps\|AA^T \|/2]
\end{eqnarray}
where we define $X_i = (Z_i - AA^T)/2$. To apply the matrix chernoff bound above we need to make sure that the variables $X_i$
meet the conditions. First, they are clearly independent since $Z_i$ are. Also, they have mean zero since $\E[Z_i] = AA^T$.
Finally, $\|X_i\| = \|(Z_i - AA^T)/2\| \le \|Z_i\|/2 + \|AA^T\|/2 \le 1$.
Thus, to apply the bound above we only need to compute $\sigma^2 = \sum_{i=1}^{r} \|\E [X^2_i] \|$.
\begin{eqnarray}
\sigma^2_i &\le& \|\E [X^2_i] \| \le \|\E [ (Z_i - AA^T)^2 ]\|/2 \\
&=&  \|\E [ Z^2_i - ZAA^T -  AA^TZ + (AA^T)^2]\|/2 \\
&=& \| AA^T - (AA^T)^2 \|/2 \le \|AA^T\|/2
\end{eqnarray}
This gives that $\sigma^2 \le r\|AA^T\|/2$.
\[
 \Pr[ \| \sum_{i=1}^{r}X_i \| > r\eps\|AA^T\|/2] \le m \cdot e^{- \frac{r\eps^2\|AA^T\|}{8}}
\]
This gives us an $\eps$ approximation in the spectral norm with probability at least $1-\delta$ if $r \ge \frac{8 }{\|AA^T\| \eps^2}\log(m/\delta)$.
Another trivial observation is that $1 = \|A\|_{fro}=tr(AA^T) \le m \|AA^T\|$  which gives that $\frac{1}{\|AA^T\|} \le m$.
To recap, for any matrix, sampling $r = \frac{8m }{\eps^2}\log(m/\delta)$ columns is sufficient in order to approximate
 $AA^T$ in the $2$-norm up to multiplicative factor $\eps\|AA^T\|$.



\section{Rank-k Approximation}
What does this tell us about the SVD. Note that the matrix resulting from the sampling above can be thought 
of the matrix $\hat{A}\hat{A}^T$ where $\hat{A} \in \R^{m\times r}$ contains rescaled sampled columns of $A$.
More accurately, $\hat{A}_{(i)} = \frac{1}{\sqrt{r}\|A_{(j)}\|}A_{j}$ if in step $i$ we picked column $j$ from $A$.

We want to say that $\hat{A}$ somehow represents $A$ well. One way to say this is that the left singular vectors of $\hat{A}$
and $A$ are ``similar'' (the right singular vectors are not in the same dimension) 
To make this more accurate we recap the property of the best rank-k approximation of $A$
\[
\|A - P_k A\| = \sigma_{k+1}
\]
Where the projection matrix $P_k = U_{k}U_{k}^{T}$ contains the top $k$ left singular vectors of $A$.
Now consider projecting $A$ on the top left singular vectors of $\hat{A}$ instead, how much do we ``loose" by that?

A lemma 4 from \cite{Drineas03passefficient} makes this exact.

\begin{lemma}
Let $\hat{P}_k$ be the projection on the top $k$ left singular vectors of $\hat{A}$, then
\[
\|A - \hat{P}_k A\|^2 \le \sigma^2_{k+1} + 2\|\hat{A}\hat{A}^T -  AA^{T}\|
\]
\end{lemma}
\begin{proof}
To see this lets compute the supremum over values $\|x(A - \hat{P}_k A)\|$, clearly $x$ is such that $x\hat{P}_k = 0$.
\begin{eqnarray}
\|A - \hat{P}_k A\|^2 &=& \langle AA^T x,x\rangle \\
&=& \langle (AA^T - \hat{A}\hat{A}^T) x,x \rangle + \langle \hat{A}\hat{A}^Tx, x\rangle \\
&\le& ||AA^T - \hat{A}\hat{A}^T|| + \hat{\sigma}^2_{k+1}
\end{eqnarray}
Where $\hat{\sigma}_{k+1}$ is the $k+1$'th singular value of $\hat{A}$.
Since, $\hat{\sigma}^{2}_{k+1} \le  \sigma^{2}_{k+1} +  ||AA^T - \hat{A}\hat{A}^T||$ we get the lemma.
\end{proof}

Finally, the SVD of $\hat{A}$ is a good approximation to the SVD of $A$ in the sense that 
\[
\|A - \hat{P}_k A\| \le \sigma_{k+1} + 2\eps\|A\|_2
\]



\bibliographystyle{unsrt}
\bibliography{AlgorithmsInDataMining.bib}

\end{document}













%%%%%%%%
