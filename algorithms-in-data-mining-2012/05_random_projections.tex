\documentclass{article}
\usepackage{algorithms_in_data_mining}

\begin{document} %
\lecture{5}{Random Projections}{Edo Liberty}

We will give a simple proof of the following, rather amazing, fact. Every set of $n$ points 
in a Euclidian space (say in dimension $d$) can be embedded into the Euclidian space of 
dimension $k = O(\log(n)/\eps^2)$ such that all pairwise distances are preserved up distortion $1\pm \eps$.
We will follow ideas from \cite{JL84} and \cite{DasGuptaGupta99} and later improve on running time using 
methods introduced in \cite{AilonCh06}.

\section*{Random projection}
We will argue that a certain distribution over the choice of a matrix $R \in \R^{k \times d}$ gives that:
\begin{equation}
\label{e1}
\forall x \in \R^{d} \;\; \Pr\left[ \left| ||\frac{1}{\sqrt{k}}Rx|| - ||x|| \right| > \eps||x|| \right] \le \frac{1}{n^2} 
\end{equation}
Before we pick this distribution and show that Equation~\ref{e1} holds for it, let us first see
that this gives the opening statement. 

Consider a set of $n$ points $x_1,\ldots, x_n$ in Euclidian space $\R^d$. Embedding these points
into a lower dimension while preserving all distances between
them up to distortion $1\pm \eps$ means approximately preserving the norms of all 
${n \choose 2}$ vectors $x_i - x_j$. Assuming Equation~\ref{e1} holds and using the union 
bound, this property will fail to hold for at least one $x_i - x_j$ pair with probability at most ${n \choose 2}\frac{1}{n^2} \le 1/2$.
Which means that all ${n \choose 2}$ point distances are preserved up to distortion $\eps$ with probability at least $1/2$.


\section{i.i.d. gaussian distribution}
We consider the distribution of matrices $R$ such that each $R(i,j)$ is drawn independently from  a
normal distribution with mean zero and variance $1$, $R(i,j) \sim \N(0,1)$. We show that for this
distribution Equation~\ref{e1} holds for some $k \in O(\log(n)/\eps^2)$.

First consider the random variable $z = \sum_{i=1}^{d}r(i)x(i)$ where $r(i) \sim \N(0,1)$. 
To understand how the variable $z$ distributes we recall the two-stability of the
normal distribution. Namely, if $z_3 = z_2 + z_1$ and 
$z_1 \sim \N(\mu_1,\sigma_{1})$ and $z_2 \sim \N(\mu_2,\sigma_{2})$ then, $$z_3 \sim \N(\mu_1 + \mu_2,\sqrt{\sigma^{2}_{1} + \sigma^{2}_{2}}).$$
In our case,  $r(i)x(i) \sim \N(0,x_{i})$ and therefore, $z = \sum_{i=1}^{d}r(i)x(i) \sim \N(0,\sqrt{\sum_{i=1}^{d}x^{2}_{i}}) \sim \N(0,||x||) \sim  ||x||y$ 
where $y_i \sim \N(0,1)$.
%
Now, note that each element in the vector $Rx$ distributes exactly like $z$.
Defining $k$ identical copies of $z$, $z_1,\ldots,z_k$,
We get that $||\frac{1}{\sqrt{k}}Rx||$ distributes exactly like:
\[
||\frac{1}{\sqrt{k}}Rx|| \sim \sqrt{\frac{1}{k}\sum_{i=1}^{k} z^{2}_{i}} \sim ||x||\sqrt{\frac{1}{k}\sum_{i=1}^{k} y^{2}_{i}}
\]
%
Thus, proving Equation~\ref{e1} reduces to showing that:
\begin{equation}
\Pr\left[ \left| \sqrt{\frac{1}{k}\sum_{i=1}^{k} y^{2}_{i}} - 1 \right| > \eps \right] \le \frac{1}{n^2} 
\end{equation}

The sum of $k$ squared normal variables is a very known distribution called chi-square with $k$ degrees of freedom, denoted by $\chi^2_k$.
It is exactly defined by $\chi^2_k = \sum_{i=1}^{k} y^{2}_{i}$ where $y_i \sim \N(0,1)$.
Since $\chi^2_k$ is a sum of independent random variables, due to the central limit theorem, $\chi^2_k$ converges to a normally distributed quantity as $k$ grows. We will use here a slightly different property $\sqrt{\chi^2_k} \sim_{k \rightarrow \infty} \N(\sqrt{k},1/\sqrt{2})$. 
Somewhat sloppily, we will assume that $k$ is large enough so that it is harmless to substitute:
$$\sqrt{\chi^2_k} \sim \N(\sqrt{k},1/\sqrt{2})$$
%
I that case we have$ \sqrt{\frac{1}{k}\sum_{i=1}^{k} y^{2}_{i}} -1\sim \N(0,\frac{1}{\sqrt{2k}})$.
Thus, we only need to show that for a random variable $Z \sim \sqrt{2k}\left[\sqrt{\frac{1}{k}\sum_{i=1}^{k} y^{2}_{i}} -1\right]  \sim \N(0,1)$ it holds that  
\begin{equation}
\Pr\left[ | Z | > \eps \sqrt{2k}\right] \le \frac{1}{n^2} 
\end{equation}
We now use a simple bound on the error function
$$\Pr[ Z  > t] =  \int_{t}^{\infty} \frac{1}{\sqrt{2\pi}}e^{-z^2/2}dz < \int_{t}^{\infty} \frac{1}{\sqrt{2\pi}}\frac{z}{t} e^{-z^2/2}dz  = \frac{1}{\sqrt{2\pi}}e^{-t^{2}/2}$$

Setting $t=\eps \sqrt{2k}$ and noting that $\Pr[Z  > t] = \Pr[Z  < -t] $ we demand that $\frac{1}{\sqrt{2\pi}}e^{-\eps^2k} \le \frac{1}{2n^2}$. 
This yields the bound $k \ge \frac{2\log(n)+O(1)}{\eps^2}$ which completes the proof.




%%%%%%%%%%%%%%%%%%%%%%%%%%%%%%%%%%%%%%%%%%%%%%%%%%%%%%
%%%%%%%%%%%%%%%%%%%%%%%%%%%%%%%%%%%%%%%%%%%%%%%%%%%%%%
%%%%%%%%%%%%%%%%%%%%%%%%%%%%%%%%%%%%%%%%%%%%%%%%%%%%%%
%%%%%%%%%%%%%%%%%%%%%%%%%%%%%%%%%%%%%%%%%%%%%%%%%%%%%%

\section{Sparse Random Projections}
The goal of this section is 
The matrix $R$ will contain a non zero only w.p. $q$. That is, 
$R(i,j) = N(0,1/\sqrt{q})$ with probability $q$ and zero else.
Again, we define $y_i = \sum_{j}R(i,j) x_j = \sum_{j}b_{i,j} g_{i,j} x_j$ where $b_{i,j} = 1$ w.p. $q$ and $g_{i,j} \sim \N(0,1/\sqrt{q})$.
For simplicity and w.o.l.g. we set $\|x\|_2 = 1$.

First notice that $\E[y^2_i]  = 1$. Also, given the values of $b_{i,j}$ we have that $y_i$ is Gaussian.
More accurately, $\sigma_{i}^2 = \sum_{j}  b_{i,j} x_{j}^{2}/q$ 

First, let us see that this is sufficient in some sense
\begin{eqnarray}
\Pr[  \frac{1}{k} \sum_{i} y_{i}^{2} \ge (1+\eps)] &= &\Pr[  e^{\lambda\sum_{i} y_{i}^{2}} \ge e^{\lambda k(1+\eps)}] \\
&\le & e^{-\lambda k(1+\eps)}\Pi_{i}\E [e^{\lambda y_{i}^{2}}]
\end{eqnarray}

Given $\sigma_i$ we have that $y_{i}$ is Gaussian and so we can compute $\E [e^{y_{i}^{2}}]$ exaclty.
\begin{eqnarray}
\E [e^{\lambda y_{i}^{2}}] &=& \frac{1}{\sqrt{2\pi \sigma^2_i}}\int_{\infty}^{\infty} e^{-\frac{y_{i}^{2}}{2\sigma^{2}_{i}}} e^{\lambda y_{i}^{2}} dy\\
&=& \frac{1}{\sqrt{2\pi\sigma^2_i}}\int_{\infty}^{\infty} e^{-(\frac{1}{\sigma^2_i}- 2\lambda)\frac{y_{i}^{2}}{2}} dy\\
&=& \frac{1}{\sqrt{1-2\lambda \sigma^{2}_{i}}}
\end{eqnarray}
Note that we must enforce now that $2\lambda \sigma^2_i < 1$.
Plugging this back into our formula we get 
\begin{eqnarray}
\Pr &\le& e^{-\lambda k(1+\eps)}\Pi_{i}\frac{1}{\sqrt{1-2\lambda \sigma^{2}_{i}}}\\
&= &e^{-\lambda k(1+\eps)  + \frac{1}{2} \sum_{i}\log(\frac{1}{1-2\lambda \sigma^{2}_{i}})   }
\end{eqnarray}
We now use the Tailor expansion by $log(\frac{1}{1-x}) \ge x + x^2$
\begin{eqnarray}
\Pr &\le& e^{-\lambda k(1+\eps)  + \frac{1}{2} \sum_{i} 2\lambda \sigma^{2}_{i}   + 4\lambda^2 \sigma^{4}_{i}}\\
&\le& e^{ \lambda (\sum_{i} \sigma^{2}_{i} - k) - \lambda k \eps  + \sum_{i} 2\lambda^2 \sigma^{4}_{i}}\\
\end{eqnarray}

Now, assume that $\sigma^2_i \le 1+\eps/2$ (we will fix this soon) then
\begin{eqnarray}
\Pr &\le& e^{ - \lambda k \eps/2  + 2\lambda^2 \sum_{i}\sigma^{4}_{i}}\\
&\le&e^{ - \frac{1}{32} \frac{k^2\eps^2}{\sum_{i}\sigma_i^4}} \le e^{-ck\eps^2}
\end{eqnarray}
for some constant $c$. As before, invoking the union bound completes the proof for some $k \in O(\log(n)/\eps^2)$.


Alas, we are left to show that $\sigma^2_i \le 1+\eps/2$. This is where the bounds on $q$ come in.
We will see that this is not true for every vector $x$ and every value of $q$.
Never the less, we'll be able to fix this and still gain on running time.
Let us recap , $\sigma^2_i = \sum_{j} b_{i,j} x_j^2/q$ where $b_{i,j} = 1$ w.p. $q$ and zero otherwise. 
Take for example $x = [1,0,\ldots,0]$. In this case $\sigma^2_i = b_{i,1}/q$ which is potentially $1/q$
which is significantly more that $1+\eps/2$. On the other hand, consider the vector $x = [\frac{1}{\sqrt{d}},\dots,\frac{1}{\sqrt{d}}]$.
In this case $\sigma^2_i =\frac{1}{d} \sum_{j} b_{i,j}/q$ whose expectation is $1$ 
and which we expect from Chernoff's inequality to be less that $1+\eps/2$ w.h.p.

Let us restrict our selves to vectors such that $\|x\|_\infty \le \eta$. I claim that the ``worst" vectors we can 
have of this form contain $1/\eta^2$ entries of value $\eta$ and the rest zeros. This is a result 
of the convexity of the moment generating functions of $\sigma^2_i$ with respect to $x$ and the fact
that the set of possible values for $\|x\|_\infty \le \eta$ lies in a polytop. 
Hence, the maximal value is attained in an extreme point as above.
Computing for this vector we have $\sigma^2_i = \sum_{j=1}^{1/\eta^2} b_{i,j}\eta^2/q$.
Bounding $\sigma^2_i$ by $1+\eps$ we get
\begin{eqnarray}
\Pr[\sigma^2_i \ge 1+\eps] &=& \Pr[\sum_{j=1}^{1/\eta^2} b_{i,j} - \frac{q}{\eta^2} \ge \frac{q\eps}{\eta^2}]\\
&\le& e^{-\frac{q\eps^2}{2\eta^2}} \le \frac{1}{cnd} \mbox{ \;\;(fail w.p. at most $1/5$)}\\
&\mbox{for}& q \ge \frac{3\log(n)\eta^2}{\eps^2}
\end{eqnarray}
Thus, if our vectors are ``spread" such that $\|x\|_\infty \le \eta < \frac{\eps}{\sqrt{3\log(n)}}$ we can save one computation and storage
by being able to set $q < 1$. 


\section{Fast Vector Spreading}


The question is, can we actively make sure that $\|x\|_\infty$ is low.
The answer is yes and a method for doing that was suggested in $\cite{AilonCh06}$.
For this we will need to learn what Hadamard matrices are.
Hadamard matrices are commonly used in coding theory and are conceptually
close for Fourier matrices. We assume for convenience that $d$ is a power of $2$.
The Walsh Hadamard transform of a vector $x \in \R^{d}$ is the
result of the matrix-vector multiplication $Hx$ where $H$ is a $d
\times d$ matrix whose entries are $H(i,j) = \frac{1}{\sqrt{d}}(-1)^{\langle
i,j\rangle}$. Here ${\langle i,j\rangle}$ means the dot product over
$F_2$ of the bit representation of $i$ and $j$ as binary vectors of
length $\log(d)$.
Another way to view this is to define Hadamard Matrices recursively.
\begin{equation*} %
H_{1} = \frac{1}{\sqrt{2}}\left(
          \begin{array}{rr}
            1 & 1 \\
            1 & -1\\
          \end{array}
        \right)
,\;\;
        H_{d} = \frac{1}{\sqrt{2}}\left(
          \begin{array}{r:r}
            H_{d/2} & H_{d/2} \\ \hdashline
            H_{d/2} & -H_{d/2}\\
          \end{array}
        \right)
\end{equation*} %
Here are a few interesting (and easy to show) facts about Hadamard matrices.
\begin{enumerate}
\item $H_d$ is a unitary matrix $\|Hx\| = \|x\|$ for any vector $x\in \R^d$.
%\item $H_{d}(i,j) \in \{ \frac{1}{\sqrt{d}},- \frac{1}{\sqrt{d}}\}$
\item Computing $x \mapsto Hx$ requires $O(d\log(d))$ operations.
\end{enumerate}


We also define a diagonal matrix $D$ to be such that $D(i,i) \in \{1,-1\}$ uniformly.
Clearly, we have that $\|HDx\|_2 = \|x\|_2$ since both $H$ and $D$ are isotropies.
Let us now bound $\|HDx\|_\infty$.
$(HDx)(1) = \sum_{i=1}^{d}H(1,i)D(i,i) x_i = \sum_{i=1}^{d}\frac{x_i}{\sqrt{d}}s_i$ where $s_i \in \{-1,1\}$ uniformly.
To bound this we recap Hoeffding's inequality.
\begin{fact}[Hoeffding's inequality]
Let $X_1,\ldots,X_n$ be independent random variables s.t. $X_i \in [a_i,b_i]$.
Let $X = \sum_{i=1}^{n} X_i$.
\begin{equation}
\Pr[|X - \E[X]| \ge t] \le 2e^{-\frac{2 t^2}{\sum_{i=1}^{n} (b_i -a_i)^2}}
\end{equation}
\end{fact}
Invoking Hoeffding's inequality and then the union bound we get that if $\|HDx\|_\infty \le \sqrt{\frac{c \log(n)}{d}}$ for all points $x$.
Remark, for this we assumed $\log(d) = O(\log(n))$ otherwise we should have had $\log(nd)$ in the bound. 
The situation, however, that the dimension is super polynomial in the number of points is unlikely. 
Usually it is common to have $n > d$.


\section{Fast Random Projecton}

Combining fast spreading with sparse projections  we get the result in \cite{AilonCh06}.
Randomly project vectors by $x \mapsto \frac{1}{\sqrt{k}}RHDx$.
Computing $HDx$ requires $O(d\log(d))$ operations and guaranties that $\|HDx\|_\infty \le \eta = \sqrt{\frac{c\log(n)}{d}}$.
Setting this into the bound on $q \ge \frac{3\log(n)\eta^2}{\eps^2}$ we get that is is sufficient to 
have $q \ge \frac{c\log^2(n)}{d\eps^2}$. The expected number of non zeros in $R$ is $qkd$.
Therefore, the expected running time required to compute $x' \mapsto Rx'$ is bounded from above by
$O(ck\log^2(n)/\eps^2) = O(\eps^2 k^3)$.
Putting this together we get a total running time of $O(d\log(d) + \eps^2 k^3)$ instead of the straight forward $O(kd)$. 

\bibliographystyle{unsrt}
\bibliography{algorithms_in_data_mining.bib}

\end{document}

