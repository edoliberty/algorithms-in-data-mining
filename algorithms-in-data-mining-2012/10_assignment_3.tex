\documentclass{article}


\documentclass{article}
\usepackage{fullpage}

%\setlength{\oddsidemargin}{0.25 in}
%\setlength{\evensidemargin}{-0.25 in}
%\setlength{\topmargin}{-0.6 in}
%\setlength{\textwidth}{6.5 in}
%\setlength{\textheight}{8.5 in}
%\setlength{\headsep}{0.75 in}
%\setlength{\parindent}{0 in}
%\setlength{\parskip}{0.1 in}



%%%% PACKAGES %%%%
\usepackage{algorithm,algorithmic}
\usepackage{amsfonts, amsmath, amsthm}
\usepackage{graphicx}
%for dotted lines inside matrices
\usepackage{arydshln}
\setlength{\dashlinedash}{.8pt} %
\setlength{\dashlinegap}{1.2pt}
%

%%%% ENVIRONMENTS %%%%
\newtheorem{definition}{Definition}[section]
\newtheorem{fact}{Fact}[section]
\newtheorem{claim}{Claim}[section]
\newtheorem{lemma}{Lemma}[section]
\newtheorem{remark}{Remark}[section]
\newtheorem{theorem}{Theorem}[section]
\newtheorem{proposition}{Proposition}[section]

%%%% COMMANDS %%%%
\newcommand{\E}{{\mathbb E}}
\newcommand{\Var}{{\operatorname{Var}}}
\newcommand{\var}{{\operatorname{Var}}}
\newcommand{\poly}{{\operatorname{poly}}}
\newcommand{\const}{{\operatorname{const}}}

\newcommand{\allones}{\mathbf{1}}
\newcommand{\abs}[1]{\left| #1 \right|}
\newcommand{\norm}[1]{\| #1 \|}
\newcommand{\eps}{\varepsilon}
\newcommand{\tab}{\hspace{.5cm}}
\newcommand{\R}{{\mathbb{R}}}
\newcommand{\Sph}{{\mathbb{S}}}
\newcommand{\N}{{\mathcal{N}}}


\newcommand{\lecture}[3]{
   %\pagestyle{myheadings}
   %\thispagestyle{plain}
   %\newpage
   %\setcounter{lecnum}{#1}
   %\setcounter{page}{1}
   \noindent
   \begin{center}
   \framebox{
      \vbox{\vspace{2mm}
    \hbox to 6.28in { {\bf 0368-3248-01-Algorithms in Data Mining
		\hfill Fall 2013} }
       \vspace{4mm}
       \hbox to 6.28in { {\Large \hfill Lecture #1: #2  \hfill} }
       \vspace{2mm}
       \hbox to 6.28in { {\it Lecturer: #3 \hfill}}% \hfill Scribes: #4} }
      \vspace{2mm}}
   }
   \end{center}
   \markboth{Lecture #1: #2}{Lecture #1: #2}
{\small
   {\bf Warning}: {\it This note may contain typos and other inaccuracies which are usually discussed during class. 
   Please do not cite this note as a reliable source. If you find mistakes, please inform me.}
   \vspace*{4mm}}
   \hrule
   \vspace{1cm}
}

%%%% COMMON DEFS %%%%
%\author{Edo Liberty \\ Class notes: Algorithms in Data mining}
\date{\nonumber}



\title{Assignment 3} %

\begin{document} %
\date{\nonumber}
\maketitle

\section{Randomized meta-algorithms}
\subsection*{setup}
In this question we assume the common case where we have an input $x \in X$  
and we wish to approximate a function $f:X \rightarrow \R^+$ (i.e. $\forall x\;\;f(x) \ge 0$).
For that we have a black box randomized algorithm $A:X\rightarrow \R^+$ such that $\E[A(x)] = f(x)$.
The questions ask you to designing meta algorithms using $A$ as a black box. 
\subsection*{question}
\begin{enumerate}
\item Show that
\[
\Pr[A(x) \ge 3f(x)] \le \frac{1}{3}
\]
\item Assume that for all $x$ we have that $\Var[A(x)] \le c\cdot [f(x)]^2$.
Describe an algorithm $B_2$ such that for any two constants $\eps,\delta > 0$:
\[
\Pr[|B_2(x) - f(x)| \ge \eps f(x)] \le \delta
\]
\item Assume that $\Pr[|A(x) - f(x) | \le t|] \ge \frac{1}{2}+\eta$ for some fixed value $\eta > 0$.
In words, the algorithm gets an additive approximation $t$ with probability slightly better than $1/2$.
(Here we do not assume anything on the variance of $A(x)$).
Design and algorithm $B_3$ such that for any prescribed $\delta >0$
\[
\Pr[|B_3(x) - f(x) | \le t|] \ge 1 - \delta
\]
That means the algorithm achieves the same additive approximation with probability arbitrary close to one.
\end{enumerate}

\pagebreak
\section{SVD and the power method}
\subsection*{setup}

Here we will prove some basic facts about singular values, matrices, and the power method.
For the reminder of the question we assume $A \in \R^{m \times n}$ is an arbitrary matrix.
For convenience and w.l.o.g. assume $m \le n$. Also, denote by $\sigma_1 \ge \ldots \sigma_m \ge 0$
the singular values of $A$.
\subsection*{question}

\begin{enumerate}
\item Let $P \in \R^{m \times m}$ and $Q \in \R^{n \times n}$ be unitary matrices.
Show that $\|PAQ\|_{fro} = \|A\|_{fro}$.
Hint, begin with the case where one of the matrices $P$ or $Q$ are the identity matrix.
\item Using the above show that for any matrix $A$ we have that 
\[
\|A\|_{fro} = \sqrt{\sum_{i=1}^{m}\sigma_{i}^{2}}.
\]
It might help you to show that $\|A\|^{2}_{fro} = tr(AA^T)$ where $tr(\cdot)$ stands for the matrix trace.
\item The numerical rank of a matrix $\rho(A) = \frac{\|A\|^{2}_{fro}}{\|A\|^2_2}$ is a smoothed
version of the algebraic rank $rank(A)$. It is always true that $1\le \rho(A) \le Rank(A) \le \min(m,n)$.
If $\rho(A) \le 1 +\eps$ for a sufficiently small $\eps$ the matrix is ``close'' to being of rank $1$.
Give an expression to the numerical rank of $A$ in terms of its
singular values $\sigma_i$. Express the numerical rank of $(AA^T)^{k}A$ in term of  $\sigma_i$.
\item Assume that the matrix $A$ is such that $\sigma_2/\sigma_1 \le \eta$ for some $\eta < 1$.
Use your expressions from above to find $k$ such that $\rho((AA^T)^{k}A)) \le  1+ \eps$.
How does this relate to the the Power Method for computing the largest singular value and vectors of $A$?
\end{enumerate}





\end{document}
