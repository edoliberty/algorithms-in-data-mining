\documentclass{article}
\usepackage{algorithms_in_data_mining}

\begin{document} %
\lecture{13}{Algorithms in Data Mining - Exam Answers}{Edo Liberty}

%\maketitle
\section*{General Info}
\begin{enumerate}
\item Solve $3$ out of $4$ questions.
\item Each correct answer is worth $33.3$ points.
\item If you have solved more than three questions, please indicate which three you would like to be checked.
\item The exam's duration is 3 hours. If you need more time please ask the attending professor.
\item Good luck!
\end{enumerate}


\section*{Useful facts}
\begin{enumerate}
\item For any vector $x \in \R^{d}$ we define the $p$-norm of $x$ as
follows:
\[
||x||_p = [\sum_{i=1}^{d}(x(i))^p]^{1/p}
\]

\item {\bf Markov's inequality:} For any {\it non-negative} random variable
$X$:
\[
\Pr[X > t] \le E[X]/t.
\]
\item {\bf Chebyshev's inequality:} For any random variable $X$:
\[
\Pr[|X - E[X]| > t] \le \var[X]/t^2.
\]
\item {\bf Chernoff's inequality:} Let $x_1,\ldots,x_n$ be independent
$\{0,1\}$ valued random variables. Each $x_i$ takes the value $1$
with probability $p_i$ and $0$ else. Let $X = \sum_{i=1}^{n}x_i$ and
let $\mu = E[X] = \sum_{i=1}^{n}p_i$. Then:
\begin{eqnarray*}
\Pr[X > (1+\eps)\mu] &\le& e^{-\mu \eps^2/4}\\
\Pr[X < (1-\eps)\mu] &\le& e^{-\mu \eps^2/2}
\end{eqnarray*}
Or in a another convenient form:
\[
\Pr[|X - \mu| > \eps\mu] \le 2e^{-\mu \eps^2/4}
\]
%\item {\bf Hoeffding's inequality:} Let $x_1,\ldots,x_n$ be
%independent random variables taking values in $\{+1,-1\}$ each with
%probability $1/2$, then:
%\[
%\Pr[|\sum_{i=1}^{n}x_i a_i| > t] \le
%2e^{-\frac{t^2}{\sum_{i=1}^{n}a_{i}^{2}}}.
%\]
%\item For any $x \ge 2$ we have:
%\[
%e^{-1} \ge (1-\frac{1}{x})^{x} \ge \frac{2}{3}e^{-1}
%\]
\item A probability distribution $\psi$ of $z$ is such that:
\[
\forall \; c_1,c_2 \in \R \;\; \Pr[c_1 \le z \le c_2]  = \int_{c_1}^{c_2}\psi(t)dt
\]

\item For a continuous variable $z$ we have that:
\[%\begin{eqnarray*)
\E[z] = \int_{-\infty}^{\infty} f(t)\psi(t)dt \;\;\;\;\;
\Var[z] = \int_{-\infty}^{\infty} f^2(t)\psi(t)dt - (\E[z])^2
\]%\end{eqnarray*}

\end{enumerate}

\pagebreak





%%%%%%%%%%%%%%%%%%%%%%%%%%%%%%%%%%%%%%%%%%%%%%%
%%%%%%%%%%%%%%%%%%%%%%%%%%%%%%%%%%%%%%%%%%%%%%%
%%%%%%%%%%%%%%%%%%%%%%%%%%%%%%%%%%%%%%%%%%%%%%%


\section{Probabilistic inequalities}
\subsection*{setup}
In this question you will be asked to derive the three most used
probabilistic inequalities for a specific random variable. Let
$x_1,\ldots,x_n$ be independent $\{-1,1\}$ valued random variables.
Each $x_i$ takes the value $1$ with probability $1/2$ and $-1$ else.
Let $X = \sum_{i=1}^{n}x_i$.

\subsection*{questions}
\begin{enumerate}
\item Let the random variable $Y$ be defined as $Y = |X|$.
Prove that Markov's inequality holds for $Y$. Hint: note that $Y$
takes integer values. Also, there is no need to compute $\Pr[Y =
i]$.
\item Prove Chebyshev's inequality for the above random variable
$X$. You can use the fact that Markov's inequality holds for any
positive variable regardless of your success (or lack of if) in the
previous question. Hint: $\var[X] = E[(X-E[X])^2]$.
\item Argue that
\[
\Pr[X > a] = \Pr[\Pi_{i=1}^{n}e^{\lambda x_i} > e^{\lambda a}] \le
\frac{E[\Pi_{i=1}^{n}e^{\lambda x_i}]}{e^{\lambda a}}
\]
for any $\lambda \in [0,1]$. Explain each transition.
\item Argue that:
\[
\frac{E[\Pi_{i=1}^{n}e^{\lambda x_i}]}{e^{\lambda a}} =
\frac{\Pi_{i=1}^{n}E[e^{\lambda x_i}]}{e^{\lambda a}} =
\frac{(E[e^{\lambda x_1}])^n}{e^{\lambda a}}
\]
What properties of the random variables $x_i$ did you use in each
transition?
\item Conclude that $\Pr[X > a] \le e^{-\frac{a^2}{2n}}$ by
showing that:
\[
\exists \;\;\lambda\in [0,1] \;\;s.t. \;\; \frac{(E[e^{\lambda
x_1}])^n}{e^{\lambda a}} \le e^{-\frac{a^2}{2n}}
\]
Hint: For the hyperbolic cosine function we have $\cosh(x) =
\frac{1}{2}(e^{x} + e^{-x}) \le e^{x^2/2}$ for $x \in [0,1]$.
\end{enumerate}
\pagebreak


%%%%%%%%%%%%%%%%%%%%%%%%%%%%%%%%%%%%%%%%%%%%%%%
%%%%%%%%%%%%%%%%%%%%%%%%%%%%%%%%%%%%%%%%%%%%%%%
%%%%%%%%%%%%%%%%%%%%%%%%%%%%%%%%%%%%%%%%%%%%%%%



\subsection*{answers}
\begin{enumerate}
\item 
\begin{eqnarray*}
E[Y] &=& \sum_{i=0}^{n}\Pr[Y=i]\cdot i\\
&=& \sum_{i=0}^{t}\Pr[Y=i]\cdot i + \sum_{i=t+1}^{n}\Pr[Y=i]\cdot i\\
&\ge& \sum_{i=t+1}^{n}\Pr[Y=i]\cdot i \\
&\ge& \sum_{i=t+1}^{n}\Pr[Y=i]\cdot t \\
&=& t\cdot\Pr[Y > t]
\end{eqnarray*}
Therefore, $E[Y] \ge t\cdot\Pr[Y > t]$ which is Markov's inequality.
\item This is identical to the general proof of Chebyshev's inequality.
We define $Z = (X - E[X])^2$. Since $Z$ is positive we can use Markov's inequality for it and get:
\[
\Pr[|X - E[X]| > t] = \Pr[Z > t^2] \le \frac{E[Z]}{t^2} = \frac{\var[X]}{t^2} 
\]
Here we used that $E[Z] = E[(X - E[X])^2] = \var[X]$.
\item First transition:
\[
\Pr[X > a] = \Pr[\lambda X > \lambda a] = \Pr[e^{\lambda X} > e^{\lambda a}] = \Pr[e^{\lambda \sum x_i} > e^{\lambda a}] = \Pr[\Pi_{i=1}^{n}e^{\lambda x_i} > e^{\lambda a}]
\]
These hold due to the monotonicity of multiplication by a positive constant and exponentiation.
Now, using Markov's inequality on the last inequality we get:
\[
\Pr[\Pi_{i=1}^{n}e^{\lambda x_i} > e^{\lambda a}] \le \frac{E[\Pi_{i=1}^{n}e^{\lambda x_i}]}{e^{\lambda a}} 
\]
\item The first transition is true due to the independence of the variables $x_i$. This means that $e^{\lambda x_i}$ are independent.
The second transition is due to all expectations of $e^{\lambda x_i}$ being equal which stems from $x_i$ being identically distributed.
\item First, we compute the expectation of $e^{\lambda x_i}$ 
\[
E[e^{\lambda x_i}] = \frac{1}{2}e^{\lambda} + \frac{1}{2}e^{-\lambda} = \cosh(\lambda) \le e^{\lambda^2/2}
\]
From the above we have that $\Pr[X > a] \le e^{n\lambda^2/2  - \lambda a}$. Setting $\lambda = a/n$ we get 
$e^{n\lambda^2/2  - \lambda a} = e^{-\frac{a^2}{2n}}$ which concludes the proof.
\end{enumerate}
\pagebreak





%%%%%%%%%%%%%%%%%%%%%%%%%%%%%%%%%%%%%%%%%%%%%%%
%%%%%%%%%%%%%%%%%%%%%%%%%%%%%%%%%%%%%%%%%%%%%%%
%%%%%%%%%%%%%%%%%%%%%%%%%%%%%%%%%%%%%%%%%%%%%%%

\section{Integrating blackbox functions}
\subsection*{setup}
Here, we will try to write an algorithm for approximately integrating blackbox functions.
Given a function $f$, the algorithm must produce an approximation for the integral of $f$ over a given range $[a,b]$.
Alas, while it can evaluate $f(t)$ for any value of $t$, it does not have any notion of the inner workings of $f$. 
More precisely, the algorithm is given a range $[a,b] \in \R$ two parameters $\eps, \delta > 0$ and a function $f$.
It is required to produce a value $A = ALG(f,a,b,\eps,\delta)$ such that with probability at least $1 -\delta$:
\[
 (1-\eps) \int_{a}^{b} f(t)dt \le A \le  (1+\eps) \int_{a}^{b} f(t)dt \ . 
\]
To make things simpler, the function $f$ is bounded both from below and from above, $\forall \; x \; 1 \le f(x) \le 2$.
The questions will lead you through constructing this algorithm.

\subsection*{questions}
\begin{enumerate}
\item Consider the variable $x$ taking values uniformly at random over the range $[a,b]$.
Write the equation for the probability distribution function $\psi$ of $x$. 
\item Prove that $\int_{a}^{b}f(t)dt = (b-a)\E[f(x)]$.
\item Show that $\var[f(x)] \le 3 (\E[f(x)])^2$. Hint: remember that $f(x) \in [1,2]$.
\item For an integer $s$, define $Y = \frac{1}{s}\sum_{i=1}^{s}f(x_i)$ where $x_i$ are all chosen uniformly and i.i.d. from $[a,b]$.
Compute $\E[Y]$ and show that $\var[Y] \le 3 (\E[Y])^2 /s$.
\item Compute a value for $s$ which guaranties that 
\[
\Pr[ |Y -  \E[Y] | \ge \eps \E[Y]] \le \delta \ .
\]
Describe the resulting algorithm $ALG(f,a,b,\eps,\delta)$ and argue that it meets the required conditions.
\end{enumerate}
\pagebreak

%%%%%%%%%%%%%%%%%%%%%%%%%%%%%%%%%%%%%%%%%%%%%%%
%%%%%%%%%%%%%%%%%%%%%%%%%%%%%%%%%%%%%%%%%%%%%%%
%%%%%%%%%%%%%%%%%%%%%%%%%%%%%%%%%%%%%%%%%%%%%%%


\subsection*{answers}
\begin{enumerate}
\item The function $\phi(x)$ is a piecewise constant function. $\phi(x) = 1/(b-a)$ for $a \le x \le b$ and zero else. 
\item To show this we simply compute the expectation of $f(x)$:
$$ \E[f(x)] = \int_{-\infty}^{\infty} f(t) \psi(t)dt = \int_{a}^{b} f(t) \frac{1}{b-a}dt $$
which gives $ \int_{a}^{b} f(t) dt = (b-a)\E[f(x)]$
\item We compute the variance of $f(x)$.
$$ \Var[f(x)] =  \int_{-\infty}^{\infty} f^{2}(t)\psi(t)dt - (\E[f(x)])^2 \le \int_{a}^{b} 4 \frac{1}{b-a}dt - (\E[f(x)])^2 = 4- (\E[f(x)])^2$$
Now, since $f(x) \in [1,2]$ we also have that $\E[f(x)] \ge 1$. This means that $4 \le 4(\E[f(x)])^2$ and gives that $\Var[f(x)] \le 3(\E[f(x)])^2$.
\item First, from linearity of expectation and the fact that $x_i$ are i.i.d. we have that:
$$\E[Y] = \E[\frac{1}{s}\sum_{i=1}^{s}f(x_i)] = \frac{1}{s}\sum_{i=1}^{s}\E[f(x_i)] = \frac{1}{s}\sum_{i=1}^{s}\E[f(x)] = \E[f(x)] \ .$$
Since $x_i$ are independante we have that the variance of the sum is the sum of variances.
$$\Var[Y] = \frac{1}{s^2}\sum_{i=1}^{s} \Var[f(x_i)] \le \frac{1}{s^2}\sum_{i=1}^{s} 3(\E[Y])^2 = 3(\E[Y])^2/s \ .$$
\item Using Chebyshev's inequality we that that 
$$ \Pr[ |Y - \E[Y]| \ge \eps\E[Y] ] \le \frac{\Var[Y]}{(\eps\E[Y])^2} \le \frac{3}{\eps^2 s} \ . $$
The condition that $\frac{3}{\eps^2 s} \le \delta$ is met by $s \ge 3/\eps^2\delta$.
%
The resulting algorithm $ALG(f,a,b,\eps,\delta)$ is trivial. 
Sample $s = 3/\eps^2\delta$ different values 
$x_1, \ldots, x_s$ uniformly at random from the interval $[a,b]$ and evaluate $f(x_i)$ for each.
Return the average of the evaluations $Y = \frac{1}{s}\sum_{i=1}^{s}f(x_i)$.

\end{enumerate}
\pagebreak

%%%%%%%%%%%%%%%%%%%%%%%%%%%%%%%%%%%%%%%%%%%%%%%
%%%%%%%%%%%%%%%%%%%%%%%%%%%%%%%%%%%%%%%%%%%%%%%
%%%%%%%%%%%%%%%%%%%%%%%%%%%%%%%%%%%%%%%%%%%%%%%



\section{Matrix Sampling}
\subsection*{setup}
Consider an $m \times n$, $\{1,-1\}$ matrix $A$. More formally, $A \in \R^{m\times n}$ and $\forall \;i \in [m], j\in[n] \; A_{i,j} \in \{1,-1\}$.
In this question we will try to compute an approximation for $AA^T$ efficiently by sampling columns from $A$.
Define $n$ i.i.d. random variables $q_1,\ldots,q_n$ such that:
\begin{equation*}
q_i = \left\{ 
\begin{array}{rl}
1/\sqrt{p} & \mbox{\; w.p. \;} p \\
0 &\mbox{ otherwise}
\end{array} 
\right.
\end{equation*}
for some fixed value $p \in [0,1]$.
The sampled matrix $B$ is such that $B_{i,j} = A_{i,j}q_j$







\subsection*{questions}
\begin{enumerate}
\item What is the expected number of non zero entries in the matrix $B$?
\item Let $A_{i}$ denote the $i$'th row of $A$ and similarly $B_{i}$. Argue that $$\E[\langle B_{i_1}, B_{i_2} \rangle] =  \langle A_{i_1}, A_{i_2} \rangle \ . $$
\item Use Chernoff's inequality to bound from above the following probability:
\[
\Pr[ | \langle B_{i_1}, B_{i_1} \rangle - \langle A_{i_1}, A_{i_1} \rangle | \ge \eps n ]
\]
for a fixed $\eps \in [0,1]$. Note that $\langle A_{i_1}, A_{i_1} \rangle = n$.
\item Bound from above the following probability:
\[
\Pr[ | \langle B_{i_1}, B_{i_2} \rangle - \langle A_{i_1}, A_{i_2} \rangle | \ge \eps n ] 
\]
Hint: it is convenient to consider the sets $ J^{+} = \{ j \;| \;A_{i_1,j} A_{i_2,j} = 1\}$ \\and $ J^{-} = \{ j \;| \;A_{i_1,j} A_{i_2,j} = -1\}$
and setting $n^{+} = |J^{+}|$ and $n^{-} = |J^{-}|$.

\item Using the union bound, compute a value for $p$ which guaranties that with probability at least $1-\delta$ we have that:
$$ \forall \;\;  i_1,i_2 \in [m] \;\;   |(BB^T)_{i_1,i_2}  - (AA^T)_{i_1,i_2}| \le \eps n \ . $$
\end{enumerate}
\pagebreak


\subsection*{answers}
\begin{enumerate}
\item The expected number of non zero variables $q_i$ is $np$. Since for each of those there are $m$ non zeros in $B$ the answer is $mnp$.
\item Here we compute the expectation 
\begin{eqnarray*}
\E[\langle B_{i_1}, B_{i_2} \rangle] &=& \E[\sum_{j=1}^{n} B_{i_1}(j) B_{i_2}(j) ] = \E[\sum_{j=1}^{n} A_{i_1}(j) A_{i_2}(j) q_j^2 ]  \\
&=& \sum_{j=1}^{n} A_{i_1}(j) A_{i_2}(j) \E[q_j^2 ] = \sum_{j=1}^{n} A_{i_1}(j) A_{i_2}(j) \\ 
&=& \langle A_{i_1}, A_{i_2} \rangle 
\end{eqnarray*}
\item Note that $\langle B_{i_1}, B_{i_1} \rangle = \sum_{j=1}^{n} A_{i_1}(j) A_{i_1}(j) q_j^2 = \frac{1}{p}\sum_{j=1}^{n}b_i$ where $b_i = 1$ w.p. $p$ and zero else. 
\[
\Pr[ | \frac{1}{p}\sum_{j=1}^{n}b_i - n | \ge \eps n ] = \Pr[ | \sum_{j=1}^{n}b_i - pn | \ge p\eps n ]
\]
Since $b_i$ are independent indicator variables we use Chernoff's bound.
\[
\Pr[ | \sum_{j=1}^{n}b_i - pn | \ge \eps pn ] \le 2e^{-\frac{pn\eps^2}{4}}
\]
\item Consider the sets $ J^{+} = \{ j \;| \;A_{i_1,j} A_{i_2,j} = 1\}$ and $ J^{-} = \{ j \;| \;A_{i_1,j} A_{i_2,j} = -1\}$. 
We have that:
$$\langle B_{i_1}, B_{i_2} \rangle = \sum_{j \in J^+} q^2_j - \sum_{j \in J^-} q^2_j$$
Moreover, setting $n^{+} = |J^{+}|$ and $n^{-} = |J^{-}|$ we have $\langle A_{i_1}, A_{i_2} \rangle = n^{+}  - n^{-}$.
Therefore, 
$$\langle B_{i_1}, B_{i_2} \rangle - \langle A_{i_1}, A_{i_2} \rangle = (\sum_{j \in J^+} q^2_j - n^{+})  - (\sum_{j \in J^-} q^2_j  - n^{-})$$
Thus, to have that $|\langle B_{i_1}, B_{i_2} \rangle - \langle A_{i_1}, A_{i_2} \rangle| \le \eps n$ it suffices to have 
$|\sum_{j \in J^+} q^2_j - n^{+}| \le \eps n/2$ and $|\sum_{j \in J^-} q^2_j - n^{-}| \le \eps n/2$.
For each one of the above we can apply the Chernoff bound as above.
$$\Pr [|\sum_{j \in J^+} q^2_j - n^{+}| \ge \eps n/2] = \Pr [|\sum_{j=1}^{n^{+}} b_j - n^{+}p | \ge \eps  np/2] \le 2e^{-\frac{(\eps  np/2)^2}{4n^{+}p}} \le 2e^{-\frac{\eps^2np}{16}}$$
which uses the fact that $n^{+} \le n$.
Repeating the same exercise for $n^-$ and using the Union bound we gets: 
\[
\Pr[ | \langle B_{i_1}, B_{i_2} \rangle - \langle A_{i_1}, A_{i_2} \rangle | \ge \eps n ] \le 4 e^{-\frac{\eps^2np}{16}}
\]

\item Using the union bound on all ${m \choose 2} \le m^2$ options of choosing $i_1$ and $i_2$ we require that: 
$$ m^2   4 e^{-\frac{\eps^2np}{16}} \le \delta \ . $$
This is satisfied by $p \ge \frac{16}{4\eps^2}\log(\frac{4m^2}{\delta})$
\end{enumerate}
\pagebreak



%%%%%%%%%%%%%%%%%%%%%%%%%%%%%%%%%%%%%%%%%%%%%%%
%%%%%%%%%%%%%%%%%%%%%%%%%%%%%%%%%%%%%%%%%%%%%%%
%%%%%%%%%%%%%%%%%%%%%%%%%%%%%%%%%%%%%%%%%%%%%%%

\section{2-Means Clustering}
\subsection*{setup}
You are given $n$ points $x_1\ldots,x_n \in \R^d$ which naturally fall into two clusters.
There exist two points $y_1,y_2 \in \R^d$ such that the distance between $y_1$ and $y_2$ is $\ell$ (that is $\|y_1 - y_2\|_2 = \ell$). 
There are $n/2$ points around $y_1$ such that $\|x_i - y_1\|_2 \le 1$. 
The other $n/2$ points are around $y_2$ and $\|x_i - y_2\|_2 \le 1$.
Note that the points $y_1$ and $y_2$ are not known to you.
Reminder: the cost of $k$-means clustering is $\min_{c_1,\ldots,c_k \in \R^d} \sum_{i=1}^{n}\min_{j \in [k]} \|x_i - c_j\|^2$.
\subsection*{questions}
\begin{enumerate}
\item What is the cost of $2$-means clustering when the two chosen cluster centers are $c_1 = y_1$ and $c_2=y_2$?
\item Argue that if we pick as centers $c_1,c_2$ two points, one from each cluster, then the cost is at most $4n$.
\item Argue that if we pick as centers $c_1,c_2$ two points from the same cluster then the cost is at least $n/2(\ell -2)^2$.
\item Assume that $\ell > 5$. Describe an algorithm for finding a clustering assignment whose cost is at most $4n$ with probability at least $1- \delta$.
Your algorithm's running time dependence on the number of points $n$ must be linear.
\item Given the algorithm in the previous question, describe an algorithm for finding the {\it optimal cluster centers} with probability $1-\delta$
and prove its correctness. (note: you are not asked to recover $y_1$ and $y_2$)
\end{enumerate}
\pagebreak

\subsection*{answers}
\begin{enumerate}
\item Each point $x_i$ is of distance $1$ from its center so the total cost is $n$.
\item Each point in the data is in the same ball (of radius $1$) with either $c_1$ or $c_2$. 
Say $x_i$ is in the same cluster as $c_1$.
From the triangle inequality, we get that 
$\|x_i -c_1\| \le \|x_i -y_1\| + \|y_i -c_1\|  \le 2$. Thus,  $\|x_i -c_1\|^4 \le 4$. 
Summing over all points we get that the total cost is at most $4n$.
\item If both $c_1$ and $c_2$ are in the same cluster than all the $n/2$ points on the other cluster (the one without centers) are
at distance at least $\ell-2$ from either $c_1$ or $c_2$. Summing over their costs we get $n/2(\ell -2)^2$.

\item Since $\ell > 5$ we have that $n/2(\ell -2)^2 \ge 4n$. 
This means that any clustering whose centers are split is better than any clustering whose centers are not split. 
Note that is we pick two points at random from the data as centers $c_1$ and $c_2$ with probability $1/2$ we pick them such that they are split.
Moreover, after we pick centers we can compute the k-means cost and make sure that the cost is at most $4n$.
If we fail, we can repeat this process until we succeed. 
It is easy to see that after at most $\log(1/\delta)$ we succeed in one of the rounds with probability at least $1-\delta$.

\item First we describe the algorithm and then prove its correctness.
We perform the previous procedure and obtain two cluster centers $c_1$ and $c_2$.
Then we set $s_1$ and $s_2$ to be the sets of points whose closest center is $c_1$ and $c_2$ respectively. 
The algorithm returns $c^{*}_1 = \frac{2}{n} \sum_{i \in s_1} x_i$ and similarly $c^{*}_2 = \frac{2}{n} \sum_{i \in s_2} x_i$
In other words, the algorithm performs one iteration of Lloyd's algorithm initialized at centers $c_1,c_2$.

The correctness stems from the following facts. First, the clusters $s_1$ and $s_2$ are optimal.
This is because $s_1$ contains the $n/2$ points of distance at most $1$ to $y_1$ and $s_2$ all the rest.
It can be verified that no other clustering is better. Assume by contradiction that such a clustering exists.
Note that it can be improved by moving one of the points between the clusters which contradicts optimality.
Second, the optimal placement of centers for a given set of points is at their average. This was shown in class.

\end{enumerate}
\pagebreak

%%%%%%%%%%%%%%%%%%%%%%%%%%%%%%%%%%%%%%%%%%%%%%%
%%%%%%%%%%%%%%%%%%%%%%%%%%%%%%%%%%%%%%%%%%%%%%%
%%%%%%%%%%%%%%%%%%%%%%%%%%%%%%%%%%%%%%%%%%%%%%%

\end{document}
